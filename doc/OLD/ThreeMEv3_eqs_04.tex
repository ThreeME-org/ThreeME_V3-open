\documentclass[12pt]{article}
\usepackage{amsmath}
\usepackage{breqn}
\numberwithin{equation}{section}
\usepackage{longtable}
\usepackage{booktabs}
\begin{document}




\section{Introduction}



This appendix provides all the equations of the model. Note that there are two versions of the household block: (1) the standard version where a LES utility function is assumed for all commodities; (2) the hybrid version where transport, car, housing and energy consumption are modeled separately. \\ \\
In this appendix, $t$ is the time operator that may be omitted when no confusion arises, $e.g.$ $Z = Z_{t}$. Variables in first difference are referred as : $\varDelta\left(Z_{t}\right) = Z_{t}-Z_{t-1}$. Therefore the logarithm difference of a variable is approximativelly its growth rate: $\varDelta\left(\operatorname{log} Z_{t}\right) \approx Z_{t}/Z_{t-1}-1$. All parameters written in Greek letter are positive. $n$ as an exponent refers to the notional value of a given variable that is the optimal value desired by the maximization agent: $e.g.$ $Z^{n}$ is the notional value of variable $Z$. Because of adjustment constraint, effective values adjust slowly to their notional value.



\subsection{Sets}





\subsection{Exogenous variables}




\section{Supply-Use equilibrium}



This section provides the equations defining the supply use - equilibrium for the domestic and imported products and at the aggregate level. It also derives the GDP according to various definitions. All equations are therefore accounting identity. \\  \\
Each identity is expressed in value and in volume. The value equation defines therefore a price index. By convention, the variable $Z$ is always expressed in volume. $PZ$ is its price. Therefore the quantity expressed in value is $Z^{VAL} = PZ * Z$. In most case, values are expresed as $PZ * Z$. When there is a risk that a variable in volume is equal to zero in simulation, we shall defined value as $Z^{VAL}$ to avoid a division by zero issue. A typical exemple would be the value-added of sector $s$: $VA^{VAL}_{s} = PY_{s} \; Y_{s} - PCI_{s} \; CI_{s}$.



\subsection{Use side}





\subsubsection{Domestic and foreign equilibrium for commodities $c$ (value \& volume):}



\noindent\textbf{Market price for the domestically produced commodity $c$} \\
\begin{dmath}
PQD_{c} \; QD_{c} = PMGRD_{c} \; MGRD_{c} + PCID_{c} \; CID_{c} + PCHD_{c} \; CHD_{c} + PGD_{c} \; GD_{c} + PID_{c} \; ID_{c} + PXD_{c} \; XD_{c} + PDSD_{c} \; DSD_{c}
\end{dmath}

\noindent\textbf{Quantity of domestically produced commodity $c$ expressed at market price} \\
\begin{dmath}
QD_{c} = MGRD_{c} + CID_{c} + CHD_{c} + GD_{c} + ID_{c} + XD_{c} + DSD_{c}
\end{dmath}

\noindent\textbf{Market price for imported commodity $c$} \\
\begin{dmath}
PQM_{c} \; QM_{c} = PMGRM_{c} \; MGRM_{c} + PCIM_{c} \; CIM_{c} + PCHM_{c} \; CHM_{c} + PGM_{c} \; GM_{c} + PIM_{c} \; IM_{c} + PXM_{c} \; XM_{c} + PDSM_{c} \; DSM_{c}
\end{dmath}

\noindent\textbf{Quantity of imported commodity $c$ expressed at market price} \\
\begin{dmath}
QM_{c} = MGRM_{c} + CIM_{c} + CHM_{c} + GM_{c} + IM_{c} + XM_{c} + DSM_{c}
\end{dmath}



\subsubsection{Margins received (value \& volume):}



The margins received by commodity $m$ corresponds to the margins supplied by this commodity. By definition, the margins received is the sum of the margins paid (or used) on the commodities $c$. \\ \\
The margins paid on domestic and imported products ($MGPD_{m, c}$ and $MGPM_{m, c}$) are defined with behavorial equations in the producer block. They follow the domestic and imported production of commodity $c$ ($YQ_{c}$ and $M_{c}$) more or less proportionnally depending on the possibility of substitutions between margins. The margins paid are then aggregated to define the margins recieved, $MGR_{m}$. The latter is then disagrated between the domestic and imported margins recieved ($MGRD_{c}$ and $MGRM_{c}$). See specification in the producer block. \\

\noindent\textbf{Market price of the margins received by commodity $m$} \\
\begin{dmath}
PMGR_{m} \; MGR_{m} = \sum_{c} PMGP_{m, c} \; MGP_{m, c}
\end{dmath}

\noindent\textbf{Margins received by commodity $m$, expressed at market price} \\
\begin{dmath}
MGR_{m} = \sum_{c} MGP_{m, c}
\end{dmath}
The margins received correspond to the sum of the margins paid to commodity $m$ over all the commodities $c$ \\

\noindent\textbf{Market price of the margins received by commodity $m$ (for verification)} \\
\begin{dmath}
PMGR^{bis}_{c} \; MGR^{bis}_{c} = PMGRD_{c} \; MGRD_{c} + PMGRM_{c} \; MGRM_{c}
\end{dmath}
Same variable as $PMGR_{c}$ above to check the accounting consistency. \\

\noindent\textbf{Margins received by commodity $m$, expressed at market price (for verification)} \\
\begin{dmath}
MGR^{bis}_{c} = MGRD_{c} + MGRM_{c}
\end{dmath}
Same variable as $MGR_{c}$ above to check the accounting consistency. \\



\subsubsection{Aggregation of imports and domestic production for commodity $c$ per use, expressed at market price (value \& volume)}


This subsection aggregates imports and domestic production for commodity $c$ for various uses. It does not do it for Exports, Households and Government final consumption ($X$, $CH$ and $G$) because these aggregates are already defined in behaviour equations (see Trade international, Consumer and Governement blocks). Expressed on value, this subsection also defines the prices indexes at market price for commodity $c$ per use as a weighted average of imported and domestic production per uses: \textit{i.e.} for $Q$ (production of commodities ), $CI$ (intermediary consumption), $I$ (private investment) and $DS$ (change in inventories). \\
 \\

\noindent\textbf{Market price of the production of commodity $c$} \\
\begin{dmath}
PQ_{c} \; Q_{c} = PQD_{c} \; QD_{c} + PQM_{c} \; QM_{c}
\end{dmath}

\noindent\textbf{Production of commodity $c$, expressed at market price} \\
\begin{dmath}
Q_{c} = QD_{c} + QM_{c}
\end{dmath}

\noindent\textbf{Market price of the intermediate consumption of commodity $c$} \\
\begin{dmath}
PCI_{c} \; CI_{c} = PCID_{c} \; CID_{c} + PCIM_{c} \; CIM_{c}
\end{dmath}

\noindent\textbf{Intermediate consumption of commodity $c$, expressed at market price} \\
\begin{dmath}
CI_{c} = CID_{c} + CIM_{c}
\end{dmath}

\noindent\textbf{Market price of the investment in commodity $c$} \\
\begin{dmath}
PI_{c} \; I_{c} = PID_{c} \; ID_{c} + PIM_{c} \; IM_{c}
\end{dmath}

\noindent\textbf{Investment in commodity $c$, expressed at market price} \\
\begin{dmath}
I_{c} = ID_{c} + IM_{c}
\end{dmath}

\noindent\textbf{Market price of the change in inventories of commodity $c$} \\
\begin{dmath}
PDS_{c} \; DS_{c} = PDSD_{c} \; DSD_{c} + PDSM_{c} \; DSM_{c}
\end{dmath}

\noindent\textbf{Change in inventories of commodity $c$, expressed at market price} \\
\begin{dmath}
DS_{c} = DSD_{c} + DSM_{c}
\end{dmath}



\subsubsection{Agregation on sectors: production of commodity $c$ per use for domestic and imported products, expressed at market price (value \& volume)}


This subsection provides the aggregates for commodity $c$ for various uses, for domestic and imported products. They are calculated through the aggregation of the corresponding sectoral data on the sector index. \\

\noindent\textbf{Market price for the domestically produced commodity $c$ used as intermediary consumption} \\
\begin{dmath}
PCID_{c} \; CID_{c} = \sum_{s} PCID_{c, s} \; CID_{c, s}
\end{dmath}

\noindent\textbf{Quantity of domestically produced commodity $c$ used as intermediary consumption, expressed at market price} \\
\begin{dmath}
CID_{c} = \sum_{s} CID_{c, s}
\end{dmath}

\noindent\textbf{Market price for imported commodity $c$ used as intermediary consumption} \\
\begin{dmath}
PCIM_{c} \; CIM_{c} = \sum_{s} PCIM_{c, s} \; CIM_{c, s}
\end{dmath}

\noindent\textbf{Quantity of imported commodity $c$ used as intermediary consumption, expressed at market price} \\
\begin{dmath}
CIM_{c} = \sum_{s} CIM_{c, s}
\end{dmath}

\noindent\textbf{Market price for domestically produced commodity $c$ used as investment} \\
\begin{dmath}
PID_{c} \; ID_{c} = \sum_{s} PID_{c, s} \; ID_{c, s}
\end{dmath}

\noindent\textbf{Quantity of imported commodity $c$ used as investment, expressed at market price} \\
\begin{dmath}
ID_{c} = \sum_{s} ID_{c, s}
\end{dmath}

\noindent\textbf{Market price for imported commodity $c$ used as investment} \\
\begin{dmath}
PIM_{c} \; IM_{c} = \sum_{s} PIM_{c, s} \; IM_{c, s}
\end{dmath}

\noindent\textbf{Quantity of imported commodity $c$ used as investment, expressed at market price} \\
\begin{dmath}
IM_{c} = \sum_{s} IM_{c, s}
\end{dmath}




\subsubsection{Agregation on commodities: imported, domestic and aggregate intermediate consumption and investment of sector $s$, expressed at market price (value \& volume)}


This subsection provides the intermediate consumption and investment of sector $s$ (imported, domestically produced and aggregated). They are calculated through the aggregation of the corresponding sectoral data on the commodity index. \\

\noindent\textbf{Market price of domestically produced intermediate consumption of sector $s$} \\
\begin{dmath}
PCID_{s} \; CID_{s} = \sum_{c} PCID_{c, s} \; CID_{c, s}
\end{dmath}

\noindent\textbf{Domestically produced intermediate consumption of sector $s$, expressed at basic price, expressed at market price} \\
\begin{dmath}
CID_{s} = \sum_{c} CID_{c, s}
\end{dmath}

\noindent\textbf{Market price of imported intermediate consumption of sector $s$} \\
\begin{dmath}
PCIM_{s} \; CIM_{s} = \sum_{c} PCIM_{c, s} \; CIM_{c, s}
\end{dmath}

\noindent\textbf{Imported intermediate consumption of sector $s$, expressed at market price} \\
\begin{dmath}
CIM_{s} = \sum_{c} CIM_{c, s}
\end{dmath}

\noindent\textbf{Market price of intermediate consumption of sector $s$} \\
\begin{dmath}
PCI_{s} \; CI_{s} = PCID_{s} \; CID_{s} + PCIM_{s} \; CIM_{s}
\end{dmath}

\noindent\textbf{Intermediate consumption of sector $s$, expressed at market price} \\
\begin{dmath}
CI_{s} = CID_{s} + CIM_{s}
\end{dmath}

\noindent\textbf{Market price of intermediate consumption of sector $s$ (for verification)} \\
\begin{dmath}
PCI^{bis} . CI^{bis} = \sum_{s} PCI_{s} \; CI_{s}
\end{dmath}
\noindent\textbf{Intermediate consumption of sector $s$, expressed at market price (for verification)} \\
\begin{dmath}
CI^{bis} = \sum_{s} CI_{s}
\end{dmath}

\noindent\textbf{Market price of domestically produced investment of sector $s$} \\
\begin{dmath}
PID_{s} \; ID_{s} = \sum_{c} PID_{c, s} \; ID_{c, s}
\end{dmath}

\noindent\textbf{Domestically produced investment of sector $s$, expressed at market price} \\
\begin{dmath}
ID_{s} = \sum_{c} ID_{c, s}
\end{dmath}

\noindent\textbf{Market price of imported investment of sector $s$} \\
\begin{dmath}
PIM_{s} \; IM_{s} = \sum_{c} PIM_{c, s} \; IM_{c, s}
\end{dmath}

\noindent\textbf{Imported investment of sector $s$, expressed at market price} \\
\begin{dmath}
IM_{s} = \sum_{c} IM_{c, s}
\end{dmath}

\noindent\textbf{Market price of investment of sector $s$} \\
\begin{dmath}
PI_{s} \; I_{s} = PID_{s} \; ID_{s} + PIM_{s} \; IM_{s}
\end{dmath}

\noindent\textbf{Investment of sector $s$, expressed at market price} \\
\begin{dmath}
I_{s} = ID_{s} + IM_{s}
\end{dmath}

\noindent\textbf{Market price of investment of sector $s$ (for verification)} \\
\begin{dmath}
PI^{bis} . I^{bis} = \sum_{s} PI_{s} \; I_{s}
\end{dmath}

\noindent\textbf{Investment of sector $s$, expressed at market price (for verification)} \\
\begin{dmath}
I^{bis} = \sum_{s} I_{s}
\end{dmath}



\subsubsection{Aggregation on commodities: imports and domestic aggregate production per use, expressed at market price (value \& volume)}


This subsection provides the aggregate production for various uses, for domestic and imported products. They are calculated through the aggregation of commodity $c$ production per use on the commodity index. \\

\noindent\textbf{Aggregate market price for domestically produced commodities} \\
\begin{dmath}
PQD . QD = \sum_{c} PQD_{c} \; QD_{c}
\end{dmath}

\noindent\textbf{Aggregate domestically produced commodities, expressed at market price} \\
\begin{dmath}
QD = \sum_{c} QD_{c}
\end{dmath}

\noindent\textbf{Aggregate market price for imported commodities} \\
\begin{dmath}
PQM . QM = \sum_{c} PQM_{c} \; QM_{c}
\end{dmath}

\noindent\textbf{Aggregate imported commodities, expressed at market price} \\
\begin{dmath}
QM = \sum_{c} QM_{c}
\end{dmath}

\noindent\textbf{Aggregate market price for the margins received on domestically produced commodities} \\
\begin{dmath}
PMGRD . MGRD = \sum_{c} PMGRD_{c} \; MGRD_{c}
\end{dmath}

\noindent\textbf{Aggregate margins received on domestically produced commodities, expressed at market price} \\
\begin{dmath}
MGRD = \sum_{c} MGRD_{c}
\end{dmath}

\noindent\textbf{Aggregate market price for the margins received on imported commodities} \\
\begin{dmath}
PMGRM . MGRM = \sum_{c} PMGRM_{c} \; MGRM_{c}
\end{dmath}

\noindent\textbf{Aggregate margins received on imported commodities, expressed at market price} \\
\begin{dmath}
MGRM = \sum_{c} MGRM_{c}
\end{dmath}

\noindent\textbf{Aggregate market price for domestically produced intermediate consumption} \\
\begin{dmath}
PCID . CID = \sum_{c} PCID_{c} \; CID_{c}
\end{dmath}

\noindent\textbf{Aggregate domestically produced intermediate consumption, expressed at market price} \\
\begin{dmath}
CID = \sum_{c} CID_{c}
\end{dmath}

\noindent\textbf{Aggregate market price for imported intermediate consumption} \\
\begin{dmath}
PCIM . CIM = \sum_{c} PCIM_{c} \; CIM_{c}
\end{dmath}

\noindent\textbf{ Aggregate imported intermediate consumption, expressed at market price} \\
\begin{dmath}
CIM = \sum_{c} CIM_{c}
\end{dmath}

\noindent\textbf{Aggregate market price for domestically produced households final consumption} \\
\begin{dmath}
PCHD . CHD = \sum_{c} PCHD_{c} \; CHD_{c}
\end{dmath}

\noindent\textbf{Aggregate domestically produced final consumption, expressed at market price} \\
\begin{dmath}
CHD = \sum_{c} CHD_{c}
\end{dmath}

\noindent\textbf{Aggregate market price for imported households final consumption} \\
\begin{dmath}
PCHM . CHM = \sum_{c} PCHM_{c} \; CHM_{c}
\end{dmath}

\noindent\textbf{Aggregate imported households final consumption, expressed at market price} \\
\begin{dmath}
CHM = \sum_{c} CHM_{c}
\end{dmath}

\noindent\textbf{Aggregate market price for domestically produced Government final consumption} \\
\begin{dmath}
PGD . GD = \sum_{c} PGD_{c} \; GD_{c}
\end{dmath}

\noindent\textbf{Aggregate domestically produced Government final consumption, expressed at market price} \\
\begin{dmath}
GD = \sum_{c} GD_{c}
\end{dmath}

\noindent\textbf{Aggregate market price for imported Government final consumption} \\
\begin{dmath}
PGM . GM = \sum_{c} PGM_{c} \; GM_{c}
\end{dmath}

\noindent\textbf{Aggregate imported Government final consumption, expressed at market price} \\
\begin{dmath}
GM = \sum_{c} GM_{c}
\end{dmath}

\noindent\textbf{Aggregate market price for domestically produced investment} \\
\begin{dmath}
PID . ID = \sum_{c} PID_{c} \; ID_{c}
\end{dmath}

\noindent\textbf{Aggregate domestically produced investment, expressed at market price} \\
\begin{dmath}
ID = \sum_{c} ID_{c}
\end{dmath}

\noindent\textbf{Aggregate market price for imported investment} \\
\begin{dmath}
PIM . IM = \sum_{c} PIM_{c} \; IM_{c}
\end{dmath}

\noindent\textbf{Aggregate imported investment, expressed at market price} \\
\begin{dmath}
IM = \sum_{c} IM_{c}
\end{dmath}

\noindent\textbf{Aggregate market price for domestically produced exports} \\
\begin{dmath}
PXD . XD = \sum_{c} PXD_{c} \; XD_{c}
\end{dmath}

\noindent\textbf{Aggregate domestically produced exports, expressed at market price} \\
\begin{dmath}
XD = \sum_{c} XD_{c}
\end{dmath}

\noindent\textbf{Aggregate market price for imported exports (re-exports)} \\
\begin{dmath}
PXM . XM = \sum_{c} PXM_{c} \; XM_{c}
\end{dmath}

\noindent\textbf{Aggregate imported exports (re-exports), expressed at market price} \\
\begin{dmath}
XM = \sum_{c} XM_{c}
\end{dmath}

\noindent\textbf{Aggregate market price for domestically produced change in inventories} \\
\begin{dmath}
PDSD . DSD = \sum_{c} PDSD_{c} \; DSD_{c}
\end{dmath}

\noindent\textbf{Aggregate domestically produced change in inventories, expressed at market price} \\
\begin{dmath}
DSD = \sum_{c} DSD_{c}
\end{dmath}

\noindent\textbf{Aggregate market price for imported change in inventories} \\
\begin{dmath}
PDSM . DSM = \sum_{c} PDSM_{c} \; DSM_{c}
\end{dmath}

\noindent\textbf{Aggregate imported change in inventories, expressed at market price} \\
\begin{dmath}
DSM = \sum_{c} DSM_{c}
\end{dmath}




\subsubsection{Aggregation of domestic and imported production per use, expressed at market price (value \& volume)}



This subsection provides the aggregate production for various uses by summing the corresponding domestic and imported aggregates. \\

\noindent\textbf{Aggregate market price for production} \\
\begin{dmath}
PQ . Q = PQD . QD + PQM . QM
\end{dmath}

\noindent\textbf{Aggregate production, expressed at market price} \\
\begin{dmath}
Q = QD + QM
\end{dmath}

\noindent\textbf{Aggregate market price for recieved margins} \\
\begin{dmath}
PMGR . MGR = PMGRD . MGRD + PMGRM . MGRM
\end{dmath}

\noindent\textbf{Aggregate recieved margins} \\
\begin{dmath}
MGR = MGRD + MGRM
\end{dmath}

\noindent\textbf{Aggregate market price for intermediate consumption} \\
\begin{dmath}
PCI . CI = PCID . CID + PCIM . CIM
\end{dmath}

\noindent\textbf{Aggregate intermediate consumption, expressed at market price} \\
\begin{dmath}
CI = CID + CIM
\end{dmath}

\noindent\textbf{Aggregate market price for household final (consumer price index)} \\
\begin{dmath}
PCH . CH = PCHD . CHD + PCHM . CHM
\end{dmath}

\noindent\textbf{Aggregate household final consumption, expressed at market price} \\
\begin{dmath}
CH = CHD + CHM
\end{dmath}

\noindent\textbf{Aggregate market price for Government final consumption} \\
\begin{dmath}
PG . G = PGD . GD + PGM . GM
\end{dmath}

\noindent\textbf{Aggregate Government final consumption, expressed at market price} \\
\begin{dmath}
G = GD + GM
\end{dmath}

\noindent\textbf{Aggregate market price for investment} \\
\begin{dmath}
PI . I = PID . ID + PIM . IM
\end{dmath}

\noindent\textbf{Aggregate investment, expressed at market price} \\
\begin{dmath}
I = ID + IM
\end{dmath}

\noindent\textbf{Aggregate market price for exports} \\
\begin{dmath}
PX . X = PXD . XD + PXM . XM
\end{dmath}

\noindent\textbf{Aggregate exports, expressed at market price} \\
\begin{dmath}
X = XD + XM
\end{dmath}

\noindent\textbf{Aggregate market price for change in inventories} \\
\begin{dmath}
PDS . DS = PDSD . DSD + PDSM . DSM
\end{dmath}

\noindent\textbf{Aggregate change in inventories, expressed at market price} \\
\begin{dmath}
DS = DSD + DSM
\end{dmath}

\newpage


\subsection{Supply side}





\subsubsection{Domestic and foreign equilibrium for commodities $c$ (value \& volume):}




\noindent\textbf{Production of commodity $c$, expressed at basic price} \\

\begin{dmath}
YQ_{c} \; PYQ_{c} + NTAXPD^{VAL}_{c} + PMGPD_{c} \; MGPD_{c} = PQD_{c} \; QD_{c}
\end{dmath}

\noindent\textbf{Basic price of the production of commodity $c$ (for verification)} \\
\begin{dmath}
PYQ^{bis}_{c} \; YQ_{c} + NTAXPD^{VAL}_{c} + PMGPD_{c} \; MGPD_{c} = PQD_{c} \; QD_{c}
\end{dmath}
 This price is already defined as a weighted average of the production price of the sectors producing commodity $c$ in the price block: $PYQ . YQ = \sum_{c} PYQ_{c} \; YQ_{c}$. To verify accountancy consistancy, we define it here under an alias name. \\

\noindent\textbf{Production of commodity $c$, expressed at basic price (for verification)} \\
\begin{dmath}
YQ^{bis}_{c} + NTAXPD_{c} + MGPD_{c} = QD_{c}
\end{dmath}
Same variable as $YQ_{c}$ above to check the accounting consistency. \\

\noindent\textbf{Imports of commodity $c$, expressed at basic price} \\
\begin{dmath}
M_{c} \; PM_{c} + NTAXPM^{VAL}_{c} + PMGPM_{c} \; MGPM_{c} = PQM_{c} \; QM_{c}
\end{dmath}

\noindent\textbf{Basic price of imports of commodity $c$ (for verification)} \\
\begin{dmath}
PM^{bis}_{c} \; M_{c} + NTAXPM^{VAL}_{c} + PMGPM_{c} \; MGPM_{c} = PQM_{c} \; QM_{c}
\end{dmath}
 This price is already defined in the price block as $PM_{c} = EXR . PWD_{c}$. To verify accountancy consistancy, we define it here under an alias name. \\

\noindent\textbf{Imports of commodity $c$, expressed at basic price (for verification)} \\
\begin{dmath}
M^{bis}_{c} + NTAXPM_{c} + MGPM_{c} = QM_{c}
\end{dmath}
Same variable as $M_{c}$ above to check the accounting consistency. \\



\subsubsection{Margins paid (value \& volume)}


\noindent\textbf{Price of the margins paid on domestically produced commodity $c$} \\
\begin{dmath}
PMGPD_{c} \; MGPD_{c} = \sum_{m} PMGPD_{m, c} \; MGPD_{m, c}
\end{dmath}

\noindent\textbf{Margins paid on the domestically produced commodity $c$} \\
\begin{dmath}
MGPD_{c} = \sum_{m} MGPD_{m, c}
\end{dmath}

\noindent\textbf{Price of the margins paid on imported commodity $c$} \\
\begin{dmath}
PMGPM_{c} \; MGPM_{c} = \sum_{m} PMGPM_{m, c} \; MGPM_{m, c}
\end{dmath}

\noindent\textbf{Margins paid on imported commodity $c$} \\
\begin{dmath}
MGPM_{c} = \sum_{m} MGPM_{m, c}
\end{dmath}

\noindent\textbf{Price of the margins paid to commodity $m$ on commodity $c$} \\
\begin{dmath}
PMGP_{m, c} \; MGP_{m, c} = PMGPD_{m, c} \; MGPD_{m, c} + PMGPM_{m, c} \; MGPM_{m, c}
\end{dmath}

\noindent\textbf{Margins paid to commodity $m$ on commodity $c$} \\
\begin{dmath}
MGP_{m, c} = MGPD_{m, c} + MGPM_{m, c}
\end{dmath}




\subsubsection{Aggregation on commodities: supply side aggregates (value \& volume)}



\noindent\textbf{Aggregate price of the margins paid on domestically produced commodity} \\
\begin{dmath}
PMGPD . MGPD = \sum_{c} PMGPD_{c} \; MGPD_{c}
\end{dmath}

\noindent\textbf{Margins paid on domestically produced commodities} \\
\begin{dmath}
MGPD = \sum_{c} MGPD_{c}
\end{dmath}

\noindent\textbf{Aggregate price of the margins paid on imported commodities} \\
\begin{dmath}
PMGPM . MGPM = \sum_{c} PMGPM_{c} \; MGPM_{c}
\end{dmath}

\noindent\textbf{Margins paid on imported commodities} \\
\begin{dmath}
MGPM = \sum_{c} MGPM_{c}
\end{dmath}

\noindent\textbf{Aggregate basic price of domestic production} \\
\begin{dmath}
PYQ . YQ = \sum_{c} PYQ_{c} \; YQ_{c}
\end{dmath}

\noindent\textbf{Domestic production, expressed at basic price} \\
\begin{dmath}
YQ = \sum_{c} YQ_{c}
\end{dmath}

\noindent\textbf{Aggregate basic price of imports} \\
\begin{dmath}
PM . M = \sum_{c} PM_{c} \; M_{c}
\end{dmath}

\noindent\textbf{Imports, expressed at basic price} \\
\begin{dmath}
M = \sum_{c} M_{c}
\end{dmath}




\subsubsection{Supply indicators of sector $s$ (value \& volume):}



\noindent\textbf{Production of sector $s$, expressed at basic price} \\
\begin{dmath}
Y_{s} = \sum_{c} Y_{c, s}
\end{dmath}
The production price of sector $s$ is defined in the producer block as a behavour equation. It can not therefore be defined here as an index. \\

\noindent\textbf{Value-added of sector $s$ expressed in value} \\
\begin{dmath}
VA^{VAL}_{s} = PY_{s} \; Y_{s} - PCI_{s} \; CI_{s}
\end{dmath}

\noindent\textbf{Value-added of sector $s$} \\
\begin{dmath}
VA_{s} = Y_{s} - CI_{s}
\end{dmath}

\noindent\textbf{Gross operating surplus of sector $s$ expressed in value} \\
\begin{dmath}
GOS^{VAL}_{s} = VA^{VAL}_{s} - PWAGES_{s} \; WAGES_{s} - PRSSC_{s} \; RSSC_{s} - NTAXI^{VAL}_{s}
\end{dmath}
The standard definition of the Gross Operating Surplus (GOS) generally include tax on profits. For simplicity, we assume that $NTAXI_{s}$ includes all net taxes on capital (i.e. tax on production and profits). In our definition, the tax on profit is therefore excluded from the GOS. This should be taken into account if one wants to use the GOS as a basis for the tax on profits. \\

\noindent\textbf{Gross operating surplus of sector $s$} \\
\begin{dmath}
GOS_{s} = VA_{s} - WAGES_{s} - RSSC_{s} - NTAXI_{s}
\end{dmath}

\noindent\textbf{Net operating surplus of sector $s$ expressed in value} \\
\begin{dmath}
NOS^{VAL}_{s} = GOS^{VAL}_{s} - PK_{s, t-1} \; \delta_{s} \; F_{K, s, t-1}
\end{dmath}

\noindent\textbf{Net operating surplus of sector $s$} \\
\begin{dmath}
NOS_{s} = GOS_{s} - PK_{s, t_{0}-1} \; \delta_{s} \; F_{K, s, t-1}
\end{dmath}




\subsubsection{Aggregation on sectors: supply indicators of all sectors  (value \& volume)}



\noindent\textbf{Basic price of aggregate production} \\
\begin{dmath}
PY . Y = \sum_{s} PY_{s} \; Y_{s}
\end{dmath}

\noindent\textbf{Aggregate production, expressed at basic price} \\
\begin{dmath}
Y = \sum_{s} Y_{s}
\end{dmath}

\noindent\textbf{Value-added price} \\
\begin{dmath}
PVA . VA = \sum_{s} VA^{VAL}_{s}
\end{dmath}

\noindent\textbf{Aggregate value-added} \\
\begin{dmath}
VA = \sum_{s} VA_{s}
\end{dmath}

\noindent\textbf{Gross wage index paid by sectors} \\
\begin{dmath}
PWAGES . WAGES = \sum_{s} PWAGES_{s} \; WAGES_{s}
\end{dmath}
The gross wage includes employees (but not employers)' social contribution \\


\noindent\textbf{Aggregate gross wages paid by sectors} \\
\begin{dmath}
WAGES = \sum_{s} WAGES_{s}
\end{dmath}


\noindent\textbf{Price of the aggregate gross operating surplus} \\
\begin{dmath}
PGOS . GOS = \sum_{s} GOS^{VAL}_{s}
\end{dmath}

\noindent\textbf{Aggregate gross operating surplus} \\
\begin{dmath}
GOS = \sum_{s} GOS_{s}
\end{dmath}

\noindent\textbf{Price of the aggregate net operating surplus} \\
\begin{dmath}
PNOS . NOS = \sum_{s} NOS^{VAL}_{s}
\end{dmath}

\noindent\textbf{Aggregate net operating surplus} \\
\begin{dmath}
NOS = \sum_{s} NOS_{s}
\end{dmath}



\subsection{Gross Domestic Product (GDP)}



In this subsection, GDP is calculated according to different approaches. All approaches lead to same result.



\subsubsection{Expenditure approach}



\noindent\textbf{Price of GDP (expenditure definition)} \\
\begin{dmath}
PGDP . GDP = PCH . CH + PG . G + PI . I + PX . X + PDS . DS - PM . M
\end{dmath}
According to expenditure approach, GDP is calculated as the sum of the different components in the final uses of goods and services. \\


\noindent\textbf{GDP (expenditure definition)} \\
\begin{dmath}
GDP = CH + G + I + X + DS - M
\end{dmath}

\noindent\textbf{Price of GDP of commodity $c$ (expenditure definition)} \\
\begin{dmath}
PGDP_{c} \; GDP_{c} = PCH_{c} \; CH_{c} + PG_{c} \; G_{c} + PI_{c} \; I_{c} + PX_{c} \; X_{c} + PDS_{c} \; DS_{c} - PM_{c} \; M_{c}
\end{dmath}

\noindent\textbf{GDP of commodity $c$ (expenditure definition)} \\
\begin{dmath}
GDP_{c} = CH_{c} + G_{c} + I_{c} + X_{c} + DS_{c} - M_{c}
\end{dmath}

\noindent\textbf{Price of GDP (expenditure definition, for verification)} \\
\begin{dmath}
PGDP^{bis} . GDP^{bis} = \sum_{c} PGDP_{c} \; GDP_{c}
\end{dmath}
\noindent\textbf{GDP (expenditure definition, for verification)} \\
\begin{dmath}
GDP^{bis} = \sum_{c} GDP_{c}
\end{dmath}



\subsubsection{Production approach}


\noindent\textbf{Price of GDP (production definition)} \\
\begin{dmath}
PGDP^{ter} . GDP^{ter} = PVA . VA + PNTAXP . NTAXP
\end{dmath}
According to production approach, GDP is calculated as the sum of the value added plus the total net taxes on commodities. \\

\noindent\textbf{ GDP (production definition)} \\
\begin{dmath}
GDP^{ter} = VA + NTAXP
\end{dmath}



\subsubsection{Income approach}


\noindent\textbf{Price of GDP (income definition)} \\
\begin{dmath}
PGDP4 . GDP4 = PGOS . GOS + PWAGES . WAGES + PRSSC . RSSC + NTAXI^{VAL} + PNTAXP . NTAXP
\end{dmath}
According to the income approach, GDP is calculated as the sum of all the economic incomes (from labor and capital) corrected by the social and taxes transfers. \\

\noindent\textbf{  GDP (income definition)} \\
\begin{dmath}
GDP4 = GOS + WAGES + RSSC + NTAXI + NTAXP
\end{dmath}



\section{Prices}



This file provides the equations defining the prices. \\

\noindent\textbf{Domestic production price of commodity $c$} \\
\begin{dmath}
PYQ_{c} \; YQ_{c} = \sum_{s} PY_{s} \; Y_{c, s}
\end{dmath}

\noindent\textbf{Notional production price of sector $s$} \\
\begin{dmath}
PY^{n}_{s} = CU^{n}_{s} \; \left( 1 + \mu_{s} \right)
\end{dmath}

\noindent\textbf{Notional mark-up of the sector $s$} \\
\begin{dmath}
\varDelta \left(\operatorname{log} 1 + \mu^{n}_{s}\right) = \rho^{\mu,Y} . \varDelta \left(\operatorname{log} CUR_{s}\right)
\end{dmath}

\noindent\textbf{Notional mark-up of the sector $s$ (definition 2)} \\
\begin{dmath}
\varDelta \left(\operatorname{log} 1 + \mu^{n2}_{s}\right) = \rho^{\mu,Y} . \left( \varDelta \left(\operatorname{log} Y_{s}\right) - \varDelta \left(\operatorname{log} Y_{s, t-1}\right) \right)
\end{dmath}






\noindent\textbf{Production capacity of the sector $s$} \\
\begin{dmath}
\varDelta \left(\operatorname{log} YCAP_{s}\right) = \sum_{f} \varphi_{f, s, t-1} \; \varDelta \left(\operatorname{log} F_{f, s} \; PROG_{f, s}\right) + \alpha^{YCAP,Y}_{s} \; \left( \operatorname{log} Y_{s, t-1} - \operatorname{log} YCAP_{s, t-1} \; CUR_{s, t_{0}} \right)
\end{dmath}

\noindent\textbf{Capacity Utilisation ratio of the sector $s$} \\
\begin{dmath}
CUR_{s} = \frac{Y_{s}}{YCAP_{s}}
\end{dmath}

\noindent\textbf{Average mark-up on commodity $c$} \\
\begin{dmath}
\left( 1 + \mu_{c} \right) = PYQ_{c} \; \frac{YQ_{c}}{\left( \sum_{s} CU_{s} \; Y_{c, s} \right)}
\end{dmath}

\noindent\textbf{Notional unit cost of production in sector $s$} \\
\begin{dmath}
CU^{n}_{s} \; Y_{s} = \sum_{f} C_{f, s} \; F^{n}_{f, s} + NTAXI^{VAL}_{s}
\end{dmath}
To define the notional price, it is preferable to use the notional unit cost of production instead of the effective one. This lead to a more stable dynamic and gives a better representation of anticipation. \\

\noindent\textbf{Unit cost of production in sector $s$} \\
\begin{dmath}
CU_{s} \; Y_{s} = \sum_{f} C_{f, s} \; F_{f, s} + NTAXI^{VAL}_{s}
\end{dmath}

\noindent\textbf{Labor cost in sector $s$} \\
\begin{dmath}
C_{L, s} = W_{s} \; \left( 1 + RRSSC_{s} \right)
\end{dmath}


\noindent\textbf{Capital cost in sector $s$} \\
\begin{dmath}
C_{K, s} = PK_{s} \; \left( \delta_{s} + r_{s} \right)
\end{dmath}
It is preferable to calculate the user cost of capital based on the price of capital rather than on the price of investment. Indeed the price of the average capital installed is lower than the one of investment because of inflation. Using the price of investment tend to over estimate the cost of capital because it assumes that the debt contracted to finance past investments is indexed on inflation which is not the case in reality. \\


\noindent\textbf{Price of capital in sector $s$} \\
\begin{dmath}
PK_{s} \; F_{K, s} = \left( 1 - \delta_{s} \right) \; PK_{s, t-1} \; F_{K, s, t-1} + PI_{s} \; I_{s}
\end{dmath}
The price of capital is calibrated by rewriting this equation in the long run. It is always smaller than 1 because it is calibrated as follows: \begin{center} $PK_{s} = \frac{PI_{s}*(Rdep_{s}+GR^{REAL})*(1+GR^{PRICES})}{(Rdep_{s}-1+(1+GR^{REAL})*(1+GR^{PRICES}))}$ \end{center}

\noindent\textbf{Energy costs in sector $s$} \\
\begin{dmath}
C_{E, s} = PE_{s}
\end{dmath}
In first approximation the cost of energy correspond to the energy price. However if the producer is forward looking, she will integrate the anticipation of price increase in it definition of the user cost of energy. In this case the specification becomes TO BE INTEGRATED \\

\noindent\textbf{Materials costs in sector $s$} \\
\begin{dmath}
C_{MAT, s} = PMAT_{s}
\end{dmath}



\subsubsection{Aggregate costs for capital, labor, energy and material}


\noindent\textbf{Aggregate cost of capital} \\
\begin{dmath}
C_{K} \; F_{K} = \sum_{s} C_{K, s} \; F_{K, s}
\end{dmath}

\noindent\textbf{Aggregate cost of labor} \\
\begin{dmath}
C_{L} \; F_{L} = \sum_{s} C_{L, s} \; F_{L, s}
\end{dmath}

\noindent\textbf{Aggregate cost of energy} \\
\begin{dmath}
C_{E} \; F_{E} = \sum_{s} C_{E, s} \; F_{E, s}
\end{dmath}

\noindent\textbf{Aggregate cost of materials} \\
\begin{dmath}
C_{MAT} \; F_{MAT} = \sum_{s} C_{MAT, s} \; F_{MAT, s}
\end{dmath}

\noindent\textbf{Gross wages paid by sector $s$ including employees (but not employers)' social contribution} \\
\begin{dmath}
WAGES_{s} \; PWAGES_{s} = W_{s} \; F_{L, s}
\end{dmath}
To derive the volume, we assume that the price is the consumer price. \\

\noindent\textbf{Price Index for gross wages} \\
\begin{dmath}
PWAGES_{s} = P
\end{dmath}





\subsubsection{Prices of commodity $c$ according to the different demand source}



\noindent\textbf{Price of commodity $c$ for household final consumption expenditure} \\
\begin{dmath}
PCH_{c} \; CH_{c} = PCHD_{c} \; CHD_{c} + PCHM_{c} \; CHM_{c}
\end{dmath}

\noindent\textbf{Price of commodity $c$ for government final consumption expenditure} \\
\begin{dmath}
PG_{c} \; G_{c} = PGD_{c} \; GD_{c} + PGM_{c} \; GM_{c}
\end{dmath}

\noindent\textbf{Price of commodity $c$ for exports use} \\
\begin{dmath}
PX_{c} \; X_{c} = PXD_{c} \; XD_{c} + PXM_{c} \; XM_{c}
\end{dmath}

\noindent\textbf{Price of commodity $c$ for sector $s$ for intermediary consumption use} \\
\begin{dmath}
PCI_{c, s} \; CI_{c, s} = PCID_{c, s} \; CID_{c, s} + PCIM_{c, s} \; CIM_{c, s}
\end{dmath}

\noindent\textbf{Materials price for sector $s$} \\
\begin{dmath}
PMAT_{s} \; F_{MAT, s} = \sum_{cm} PCI_{cm, s} \; CI_{cm, s}
\end{dmath}

\noindent\textbf{Energy price for sector $s$} \\
\begin{dmath}
PE_{s} \; F_{E, s} = \sum_{ce} PCI_{ce, s} \; CI_{ce, s}
\end{dmath}



\noindent\textbf{Selling price of commodity $c$} \\
\begin{dmath}
PYQS_{c} \; YQS_{c} = PYQ_{c} \; YQ_{c} + PMGPD_{c} \; MGPD_{c} + NTAXPD^{VAL}_{c}
\end{dmath}
$YQS_{c}$ is the volume of the production expressed at market price. It should not be seen as a composite of several "goods": production at base price, margins and taxes. Its does not increase when the volume of the margins and taxes increase. The price does instead. This is equivalent to assuming that $YQS_{c}$ is always proportionnal to and $YQ_{c}$ since the volume of margins and taxes depends on the latter. Writing it following the specification composite of several goods, $YQS_{c} = YQ_{c} + MGPD_{c} + NTAXPD_{c}$, would lead to inacurate results since a decrease in the quantity of margins used per unit of production would not lead to a decrease of the selling price. \\

\noindent\textbf{Quantity of domestically produced commodity $c$ expressed at selling price} \\
\begin{dmath}
\varDelta \left(\operatorname{log} YQS_{c}\right) = \varDelta \left(\operatorname{log} YQ_{c}\right)
\end{dmath}

\noindent\textbf{Selling price for imported commodity $c$} \\
\begin{dmath}
PMS_{c} \; MS_{c} = PM_{c} \; M_{c} + NTAXPM^{VAL}_{c} + PMGPM_{c} \; MGPM_{c}
\end{dmath}

\noindent\textbf{Quantity of imported commodity $c$ expressed at selling price} \\
\begin{dmath}
\varDelta \left(\operatorname{log} MS_{c}\right) = \varDelta \left(\operatorname{log} M_{c}\right)
\end{dmath}

\noindent\textbf{Price of the margins paid to commodity $m$ on domestically produced commodity $c$} \\
\begin{dmath}
PMGPD_{m, c} \; MGR_{m} = PMGRD_{m} \; MGRD_{m} + PMGRM_{m} \; MGRM_{m}
\end{dmath}
We assume that the margins paid on domestic and imported commodities can be produced by domestic and foreign (using the import share of the margin received). The price of the margins paid to commodity $m$ is assumed commun to all commodity $c$. \\

\noindent\textbf{Price of the margins paid to commodity $m$ on imported commodity $c$} \\
\begin{dmath}
PMGPM_{m, c} = PMGPD_{m, c}
\end{dmath}
This price is the same as the one paid on domestic commodity because of the assumption given in the previous equation. \\

\noindent\textbf{Price of margins received on domestically produced commodity $c$} \\
\begin{dmath}
PMGRD_{c} = PYQS_{c}
\end{dmath}

\noindent\textbf{Price of margins received on imported commodity $c$} \\
\begin{dmath}
PMGRM_{c} = PMS_{c}
\end{dmath}



\subsubsection{Price of intermediary raw material consumption domestically produced $c$ of sector $s$}



\noindent\textbf{Price of domestically produced commodity $c$ for sector $s$ for intermediate consumption use} \\
\begin{dmath}
PCID_{c, s} = PYQS_{c}
\end{dmath}

\noindent\textbf{Price of imported commodity $c$ for sector $s$ for intermediate consumption use} \\
\begin{dmath}
PCIM_{c, s} = PMS_{c}
\end{dmath}

\noindent\textbf{Price of domestically produced commodity $c$ for households final consumption expenditure} \\
\begin{dmath}
PCHD_{c} = PYQS_{c}
\end{dmath}

\noindent\textbf{Price of imported commodity $c$ for households final consumption expenditure} \\
\begin{dmath}
PCHM_{c} = PMS_{c}
\end{dmath}

\noindent\textbf{Price of domestically produced commodity $c$ for government final consumption expenditure} \\
\begin{dmath}
PGD_{c} = PYQS_{c}
\end{dmath}

\noindent\textbf{Price of imported commodity $c$ for government final consumption expenditure} \\
\begin{dmath}
PGM_{c} = PMS_{c}
\end{dmath}

\noindent\textbf{Price of domestically produced commodity $c$ for investment use} \\
\begin{dmath}
PID_{c, s} = PYQS_{c}
\end{dmath}

\noindent\textbf{Price of imported commodity $c$ for investment use} \\
\begin{dmath}
PIM_{c, s} = PMS_{c}
\end{dmath}

\noindent\textbf{Price of domestically produced commodity $c$ for export use} \\
\begin{dmath}
PXD_{c} = PYQS_{c}
\end{dmath}

\noindent\textbf{Price of imported commodity $c$ for export use} \\
\begin{dmath}
PXM_{c} = PMS_{c}
\end{dmath}

\noindent\textbf{Price of domestically produced commodity $c$ for change in inventories use} \\
\begin{dmath}
PDSD_{c} = PYQS_{c}
\end{dmath}

\noindent\textbf{Price of imported commodity $c$ for change in inventories use} \\
\begin{dmath}
PDSM_{c} = PMS_{c}
\end{dmath}

\noindent\textbf{Price of imported commodity $c$} \\
\begin{dmath}
PM_{c} = EXR . PWD_{c}
\end{dmath}

\noindent\textbf{Notional wage in sector $s$} \\
\begin{dmath}
\varDelta \left(\operatorname{log} W^{n}_{s}\right) = \rho^{W,Cons}_{s} + \rho^{W,P}_{s} \; \varDelta \left(\operatorname{log} P\right) + \rho^{W,Pe}_{s} \; \varDelta \left(\operatorname{log} P^{e}\right) + \rho^{W,PROG}_{s} \; \varDelta \left(\operatorname{log} PROG^{L}_{s}\right) - \rho^{W,U}_{s} \; \left( UnR - DNAIRU \right) - \rho^{W,DU}_{s} \; \varDelta \left(UnR\right) + \rho^{W,L}_{s} \; \varDelta \left(\operatorname{log} F_{L, s} - \operatorname{log} F_{L}\right)
\end{dmath}
In order to have a $NAIRU$ that is not predetermined, we have to assume that the constant is a function of the unemployment rate: \begin{center} $d(\rho^{W,Cons}_{s}) = 0.9 * \rho^{W,U}_{s} * d(UnR)$ \end{center}

\noindent\textbf{Average wage} \\
\begin{dmath}
W . F_{L} = \left( \sum_{s} W_{s} \; F_{L, s} \right)
\end{dmath}

\noindent\textbf{Consumer Price Index} \\
\begin{dmath}
P = PCH
\end{dmath}



\subsubsection{"Original" Taylor rule}


\noindent\textbf{Notional interest rate of the Central Bank (Taylor rule)} \\
\begin{dmath}
\varDelta \left(R^{n}\right) = \rho^{Rdir,Cons} + \rho^{Rdir,P} . \varDelta \left(\frac{\varDelta \left(P\right)}{P_{t-1}}\right) - \rho^{Rdir,UnR} . \varDelta \left(UnR\right)
\end{dmath}
This general specification combines various wage equation found in the literature: the Phillips curve and the WS curve. The WS curve à la Layard et al. (2005) requires the following constraints : \\{$\rho^{W_{P}}_{s} = \rho^{W,PROG}_{s} = 1, \rho^{W,U}_{s} = \rho^{W,Cons}_{s}= 0$} \\

\noindent\textbf{Interest rate paid on capital by sector $s$} \\
\begin{dmath}
\varDelta \left(R_{s}\right) = \varDelta \left(R\right)
\end{dmath}
We assume a constant premiun on the interest rate of the Central Bank \\

\noindent\textbf{Interest rate paid by the Governement on its debt} \\
\begin{dmath}
\varDelta \left(r^{DEBT,G}\right) = \varDelta \left(r\right)
\end{dmath}
We assume a constant premiun on the interest rate of the Central Bank \\


\section{Producer}



This file provides the equations defining the producer behaviour.
Equation are behavioral. They are not used to calibrate the initial value of variables. They may be inverted to calibrate a parameter.



\subsection{Margins}



\noindent\textbf{Margins paid to commodity $m$ on the domestic commodity $c$} \\
\begin{dmath}
\varDelta \left(\operatorname{log} MGPD_{m, c}\right) = \varDelta \left(\operatorname{log} YQ_{c}\right) + \varDelta \left(SUBST^{MGPD}_{m, c}\right)
\end{dmath}
The growth in demand for margins follows the growth of aggregate demand for the commodity $c$ and a substitution term \\

\noindent\textbf{Notional substitution between margin-making sectors $m$ for the domestically produced commodity $c$} \\
\begin{dmath}
SUBST^{n,MGPD}_{m, c} = \sum_{mm} -\sigma^{MGPD}_{m, mm, c} \; \varphi^{MGPD}_{mm, c, t-1} \; \varDelta \left(\operatorname{log} PMGPD_{m, c} - \operatorname{log} PMGPD_{mm, c}\right)
\end{dmath}

\noindent\textbf{Market share of the margin-making sector $m$  for the commodity $c$} \\
\begin{dmath}
\varphi^{MGPD}_{m, c} = PMGPD_{m, c} \; \frac{MGPD_{m, c}}{\left( \sum_{mm} PMGPD_{mm, c} \; MGPD_{mm, c} \right)}
\end{dmath}

\noindent\textbf{Margins paid to commodity $m$ on the imported commodity $c$} \\
\begin{dmath}
\varDelta \left(\operatorname{log} MGPM_{m, c}\right) = \varDelta \left(\operatorname{log} M_{c}\right) + \varDelta \left(SUBST^{MGPM}_{m, c}\right)
\end{dmath}

\noindent\textbf{Notional substitution effect between the margin-making sector $m$ and the over margin-makings sectors $mm$ for the imported commodity $c$} \\
\begin{dmath}
SUBST^{n,MGPM}_{m, c} = \sum_{mm} -\sigma^{MGPM}_{m, mm, c} \; \varphi^{MGPM}_{mm, c, t-1} \; \varDelta \left(\operatorname{log} PMGPM_{m, c} - \operatorname{log} PMGPM_{mm, c}\right)
\end{dmath}

\noindent\textbf{ share of the margin type $m$ on total margins paid on the domestic commodity $c$} \\
\begin{dmath}
\varphi^{MGPM}_{m, c} = PMGPM_{m, c} \; \frac{MGPM_{m, c}}{\left( \sum_{mm} PMGPM_{mm, c} \; MGPM_{mm, c} \right)}
\end{dmath}



\subsection{Production factors}



\noindent\textbf{Production of commodity $c$ by sector $s$} \\
\begin{dmath}
Y_{c, s} = PhiY_{c, s} \; YQ_{c}
\end{dmath}
We assume that each activity $s$ may produce more than one commodity $c$. Therefore the production $Y$ of commodity $c$ by the activity $s$ depends on the parameter $\varphi^{Y}_{c, s}$ which represents the share of sector $s$ in the total production of commodity $c$. \\

\noindent\textbf{Demand for production factor $f$ of sector $s$} \\
\begin{dmath}
\varDelta \left(\operatorname{log} F^{n}_{f, s}\right) = \varDelta \left(\operatorname{log} Y_{s}\right) - \varDelta \left(\operatorname{log} PROG_{f, s}\right) + \varDelta \left(SUBST^{F}_{f, s}\right)
\end{dmath}

\noindent\textbf{Notional substitution effect between the input $f$ and the over inputs $ff$} \\
\begin{dmath}
\varDelta \left(SUBST^{n,F}_{f, s}\right) = \sum_{ff} -ES_{f, ff, s} \; \varphi_{ff, s, t-1} \; \varDelta \left(\operatorname{log} \frac{C_{f, s}}{PROG_{f, s}} - \operatorname{log} \frac{C_{ff, s}}{PROG_{ff, s}}\right)
\end{dmath}

\noindent\textbf{Share of production factor $f$ of sector $s$} \\
\begin{dmath}
\varphi_{f, s} = \frac{C_{f, s} \; F^{n}_{f, s}}{\sum_{ff} C_{ff, s} \; F^{n}_{ff, s}}
\end{dmath}







\subsubsection{Aggregate production factors for capital, labor, energy and material}


\noindent\textbf{Aggregate capital input} \\
\begin{dmath}
F_{K} = \sum_{s} F_{K, s}
\end{dmath}

\noindent\textbf{Aggregate labor input} \\
\begin{dmath}
F_{L} = \sum_{s} F_{L, s}
\end{dmath}

\noindent\textbf{Aggregate energy input} \\
\begin{dmath}
F_{E} = \sum_{s} F_{E, s}
\end{dmath}

\noindent\textbf{Aggregate materials input} \\
\begin{dmath}
F_{MAT} = \sum_{s} F_{MAT, s}
\end{dmath}

\noindent\textbf{Investment use of commodity $c$ by sector $s$} \\
\begin{dmath}
\varDelta \left(\operatorname{log} I_{c, s}\right) = \varDelta \left(\operatorname{log} IA_{s}\right)
\end{dmath}
For a given sector, we assume that the investment structure is fixed over time. In other words, the investment good is a composite of several commodities in fixed proportion. \\

\noindent\textbf{Energy input demand by type of energy $ce$ by sector $s$} \\
\begin{dmath}
\varDelta \left(\operatorname{log} CI_{ce, s}\right) = \varDelta \left(\operatorname{log} F_{E, s}\right) + \varDelta \left(SUBST^{CI}_{ce, s}\right)
\end{dmath}

\noindent\textbf{Notional substitution effect between the energy commodity $ce$ and the over energy commodities $cee$ for the sector $s$} \\
\begin{dmath}
\varDelta \left(SUBST^{n,CI}_{ce, s}\right) = \sum_{cee} -\sigma^{NRJ}_{ce, cee, s} \; \varphi_{E, cee, s, t-1} \; \varDelta \left(\operatorname{log} PCI_{ce, s} - \operatorname{log} PCI_{cee, s}\right)
\end{dmath}

\noindent\textbf{Share of energy input $ce$  on total energy use by sector $s$} \\
\begin{dmath}
\varphi_{E, ce, s} = \frac{PCI_{ce, s} \; CI_{ce, s}}{\sum_{cee} PCI_{cee, s} \; CI_{cee, s}}
\end{dmath}

\noindent\textbf{Demand for material commodity $cmo$ by sector $s$} \\
\begin{dmath}
\varDelta \left(\operatorname{log} CI_{cmo, s}\right) = \varDelta \left(\operatorname{log} F_{MAT, s}\right)
\end{dmath}
Intermediary consumption that are not transport or energy commodities are not substitutables (Leontief technology)  \\

\noindent\textbf{Demand for transport commodities by sector $s$} \\
\begin{dmath}
\varDelta \left(\operatorname{log} TRSP_{s}\right) = \varDelta \left(\operatorname{log} F_{MAT, s}\right)
\end{dmath}


\noindent\textbf{Demand for transport commodity $ct$ by sector $s$} \\
\begin{dmath}
\varDelta \left(\operatorname{log} CI_{ct, s}\right) = \varDelta \left(\operatorname{log} TRSP_{s}\right) + \varDelta \left(SUBST^{CI}_{ct, s}\right)
\end{dmath}

\noindent\textbf{Notional substitution effect between the transport $ct$ and the over transports $mt$ for the sector $s$} \\
\begin{dmath}
\varDelta \left(SUBST^{n,CI}_{ct, s}\right) = \sum_{ctt} -\sigma^{TRSP}_{ct, ctt, s} \; \varphi^{TRSP}_{ctt, s, t-1} \; \varDelta \left(\operatorname{log} PCI_{ct, s} - \operatorname{log} PCI_{ctt, s}\right)
\end{dmath}

\noindent\textbf{Share for transport $ct$ use in total transport by sector $s$} \\
\begin{dmath}
\varphi^{TRSP}_{ct, s} = \frac{PCI_{ct, s} \; CI_{ct, s}}{\sum_{ctt} PCI_{ctt, s} \; CI_{ctt, s}}
\end{dmath}

\noindent\textbf{Technical progress of the production factor $f$ in the sector $s$} \\
\begin{dmath}
PROG_{f, s} = PROG_{f, s, t-1} \; \left( 1 + GR^{PROG}_{f, s} \right)
\end{dmath}

\noindent\textbf{Endogenous energy efficiency} \\
\begin{dmath}
GR^{PROG}_{E, s} = GR^{PROG}_{E, s, t_{0}} + \rho^{PROG,E,PE} . \left( \operatorname{log} PE_{s} - \operatorname{log} P > 0 \right) \; \varDelta \left(\operatorname{log} PE_{s} - \operatorname{log} P\right)
\end{dmath}
This specification states that the productivity gain of energy input in the sector $s$ for the energy type $ce$ depends on a steady-state trend (exogenous) and a price-induced component. This component is equal to a  $\rho^{PROG,E,PE}$ share of the log-difference between the level of the general energy price index for the sector $s$ to the general level of prices. \\








\section{Consumer}





\subsection{Households' Income}


\noindent\textbf{Disposable income before tax in value} \\
\begin{dmath}
DISPINC^{BT,VAL} = PWAGES . WAGES + PROP^{INC,H,VAL} + SOC^{BENF,VAL} + TRSF^{HH,VAL}
\end{dmath}
The disposable income before tax is used as base for the income tax. \\

\noindent\textbf{Disposable income after tax in value} \\
\begin{dmath}
DISPINC^{AT,VAL} = DISPINC^{BT,VAL} - INC^{SOC,TAX,VAL}
\end{dmath}
The definition of the disposable income after tax corresponds to the definition of "gross disposable income" defined in the annual account by institutional sector of Eurostat (b.6.g). \\

\noindent\textbf{Income \& Social Taxes in value} \\
\begin{dmath}
INC^{SOC,TAX,VAL} = RINC^{SOC,TAX} . DISPINC^{BT,VAL}
\end{dmath}

\noindent\textbf{Property incomes in value} \\
\begin{dmath}
PROP^{INC,H,VAL,n} = \varphi^{PROP^{INC,H}} . PNOS . NOS
\end{dmath}

\noindent\textbf{Social benefits in value} \\

\begin{dmath}
SOC^{BENF,VAL} = RR^{POP} . W . PROG^{L} . P . POP + RR^{Un} . W . Un
\end{dmath}









\subsection{Households' Expenditure}



\noindent\textbf{Aggregate notional households final consumption in value} \\
\begin{dmath}
CH^{n,VAL} = DISPINC^{AT,VAL} . \left( 1 - MPS^{n} \right)
\end{dmath}

\noindent\textbf{Notional marginal propensity to save} \\
\begin{dmath}
\varDelta \left(MPS^{n}\right) = \rho^{MPS,R} . \varDelta \left(R - \frac{\varDelta \left(P\right)}{P_{t-1}}\right) + \rho^{MPS,UnR} . \varDelta \left(UnR\right)
\end{dmath}

\noindent\textbf{Households  final consumption of commodity $c$} \\
\begin{dmath}
\left( CH^{n}_{c} - NCH_{c} \right) \; PCH_{c} = \varphi^{MCH}_{c} \; \left( CH^{n,VAL} - PNCH . NCH \right)
\end{dmath}

\noindent\textbf{Price of necessary households consumption of commodity $c$} \\
\begin{dmath}
PNCH . NCH = \sum_{c} PNCH_{c} \; NCH_{c}
\end{dmath}

\noindent\textbf{Necessary households final consumption of commodity $c$} \\
\begin{dmath}
NCH = \sum_{c} NCH_{c}
\end{dmath}

\noindent\textbf{Share of commodity $c$ in the marginal household consumption} \\
\begin{dmath}
\varDelta \left(\operatorname{log} \varphi^{MCH}_{c}\right) = \left( 1 - \sigma^{LESCES} \right) . \varDelta \left(\operatorname{log} \frac{PCH_{c}}{PCH^{CES}}\right)
\end{dmath}

\noindent\textbf{Share of commodity $c$ in the household consumption} \\
\begin{dmath}
\varphi^{CH}_{c} = \frac{CH_{c}}{CH}
\end{dmath}

\noindent\textbf{Consumption price} \\


\noindent\textbf{Households savings in value} \\
\begin{dmath}
SAV^{H,VAL} = DISPINC^{AT,VAL} - PCH . CH
\end{dmath}

\noindent\textbf{Households savings rate} \\
\begin{dmath}
RSAV^{H,VAL} = \frac{SAV^{H,VAL}}{DISPINC^{AT,VAL}}
\end{dmath}

\noindent\textbf{Households savings stock} \\
\begin{dmath}
Stock^{SAV,H,VAL} = Stock^{SAV,H,VAL}_{t-1} + SAV^{H,VAL}
\end{dmath}


\section{Government}




\subsection{Taxes}



We assume that the volume of the tax varies only when the volume of the taxe bases (e.g. production, consumption) varies. Hence an increase in the tax rate does not increase the volume of the tax but increases its price. This is coherent with the specification of the price of the tax bases: increasing the tax rate on production increases the production price but not production. \\

\noindent\textbf{Net taxes on domestically produced commodity $c$ in value} \\
\begin{dmath}
NTAXPD^{VAL}_{c} = RNTAXPD_{c} \; PYQ_{c} \; YQ_{c}
\end{dmath}

\noindent\textbf{Net taxes on domestically produced commodity $c$ in volume} \\
\begin{dmath}
NTAXPD_{c} = RNTAXPD_{c, t_{0}} \; YQ_{c}
\end{dmath}

\noindent\textbf{Net taxes on imported commodity $c$ in value} \\
\begin{dmath}
NTAXPM^{VAL}_{c} = RNTAXPM_{c} \; PM_{c} \; M_{c}
\end{dmath}

\noindent\textbf{Net taxes on imported commodity $c$ in volume} \\
\begin{dmath}
NTAXPM_{c} = RNTAXPM_{c, t_{0}} \; M_{c}
\end{dmath}



\noindent\textbf{Net taxes on commodity $c$ in value} \\
\begin{dmath}
NTAXP^{VAL}_{c} = NTAXPD^{VAL}_{c} + NTAXPM^{VAL}_{c}
\end{dmath}

\noindent\textbf{Net taxes on commodity $c$ in volume} \\
\begin{dmath}
NTAXP_{c} = NTAXPD_{c} + NTAXPM_{c}
\end{dmath}

\noindent\textbf{Aggregate net taxes on commodity $c$ in value} \\
\begin{dmath}
PNTAXP . NTAXP = \sum_{c} NTAXP^{VAL}_{c}
\end{dmath}

\noindent\textbf{Aggregate net taxes on commodity $c$ in volume} \\
\begin{dmath}
NTAXP = \sum_{c} NTAXP_{c}
\end{dmath}

\noindent\textbf{Net taxes on production of sector $s$ in value} \\
\begin{dmath}
NTAXI^{VAL}_{s} = RNTAXI_{s} \; PY_{s} \; Y_{s}
\end{dmath}

\noindent\textbf{Net taxes on production of sector $s$ in volume} \\
\begin{dmath}
NTAXI_{s} = RNTAXI_{s, t_{0}} \; Y_{s}
\end{dmath}

\noindent\textbf{Aggregate net taxes on production (value \& volume)} \\


\noindent\textbf{Net taxes on production in value} \\
\begin{dmath}
NTAXI^{VAL} = \sum_{s} NTAXI^{VAL}_{s}
\end{dmath}

\noindent\textbf{Net taxes on production in volume} \\
\begin{dmath}
NTAXI = \sum_{s} NTAXI_{s}
\end{dmath}

\noindent\textbf{Employers' social security contribution paid by sector $s$ expressed in consumer price} \\
\begin{dmath}
RSSC_{s} \; PRSSC_{s} = W_{s} \; F_{L, s} \; RRSSC_{s}
\end{dmath}
$RSSC$ stands for employers' Social Security Contribution \\

\noindent\textbf{Price of RSSC for sector $s$} \\
\begin{dmath}
PRSSC_{s} = P
\end{dmath}

\noindent\textbf{Total employers' social security contribution expressed in consumer price} \\
\begin{dmath}
PRSSC . RSSC = \sum_{s} PRSSC_{s} \; RSSC_{s}
\end{dmath}

\noindent\textbf{Price of RSSC} \\
\begin{dmath}
RSSC = \sum_{s} RSSC_{s}
\end{dmath}

\noindent\textbf{Average employers' social security contribution rate} \\
\begin{dmath}
RRSSC = PRSSC . \frac{RSSC}{\left( W . F_{L} \right)}
\end{dmath}

\noindent\textbf{ Government final consumption expenditure of commodity $c$} \\
\begin{dmath}
\varDelta \left(\operatorname{log} G_{c}\right) = \varDelta \left(\operatorname{log} EXPG\right)
\end{dmath}

\noindent\textbf{Notional Property incomes of the Government in value} \\
\begin{dmath}
PROP^{INC,G,VAL,n} = \varphi^{PROP^{INC,G}} . PNOS . NOS
\end{dmath}

\noindent\textbf{Incomes of the Government in value} \\
\begin{dmath}
INC^{G,VAL} = PNTAXP . NTAXP + NTAXI^{VAL} + INC^{SOC,TAX,VAL} + PRSSC . RSSC + PROP^{INC,G,VAL}
\end{dmath}

\noindent\textbf{Spendings of the Government in value} \\
\begin{dmath}
SPEND^{G,VAL} = PG . G + SOC^{BENF,VAL} + DEBT^{G,VAL}_{t-1} \; \left( \varphi^{RD^{G}}_{t-1} + r^{DEBT,G}_{t-1} \right)
\end{dmath}

\noindent\textbf{Savings of the Government in value (Net lending/borrowing: published deficit/savings of the Government)} \\
\begin{dmath}
SAV^{G,VAL} = INC^{G,VAL} - SPEND^{G,VAL}
\end{dmath}

\noindent\textbf{Primary balance of the Government in value (deficit).} \\
\begin{dmath}
Bal^{G,Prim,VAL} = SAV^{G,VAL} + DEBT^{G,VAL}_{t-1} \; \left( \varphi^{RD^{G}}_{t-1} + r^{DEBT,G}_{t-1} \right)
\end{dmath}
It corresponds to the savings excluding the reimbursement and the interest on the debt. \\

\noindent\textbf{Primary balance of the Government in value (deficit)from an alternative expression} \\
\begin{dmath}
Bal^{G,Prim,VAL,bis} = INC^{G,VAL} - \left( PG . G + SOC^{BENF,VAL} \right)
\end{dmath}
It corresponds to the saving excluding the reimbursement but not the interest on the debt. \\

\noindent\textbf{Total balance of the Government in value (deficit)} \\
\begin{dmath}
Bal^{G,Tot,VAL} = Bal^{G,Prim,VAL} - DEBT^{G,VAL}_{t-1} \; r^{DEBT,G}_{t-1}
\end{dmath}

\noindent\textbf{Government's debt in value} \\
\begin{dmath}
DEBT^{G,VAL} = DEBT^{G,VAL}_{t-1} \; \left( 1 - \varphi^{RD^{G}}_{t-1} \right) - SAV^{G,VAL}
\end{dmath}
It corresponds to the previous year debt minus the reimbursement of the debt and the government savings. \\

\noindent\textbf{Government's savings rate in value (in percent of GDP)} \\
\begin{dmath}
RSAV^{G,VAL} = \frac{SAV^{G,VAL}}{\left( PGDP . GDP \right)}
\end{dmath}

\noindent\textbf{Primary balance of the Government in value (in percent of GDP)} \\
\begin{dmath}
RBal^{G,Prim,VAL} = \frac{Bal^{G,Prim,VAL}}{\left( PGDP . GDP \right)}
\end{dmath}

\noindent\textbf{Total balance of the Government in value (in percent of GDP)} \\
\begin{dmath}
RBal^{G,Tot,VAL} = \frac{Bal^{G,Tot,VAL}}{\left( PGDP . GDP \right)}
\end{dmath}

\noindent\textbf{Ratio of the Government's debt in value (in percent of GDP)} \\
\begin{dmath}
RDEBT^{G,VAL} = \frac{DEBT^{G,VAL}}{\left( PGDP . GDP \right)}
\end{dmath}



\section{International Trade}


This file provides the equations defining the allocation between domestic and imported goods per use.The differentiation per use allows for distinguishing import share per use and therefore a more realistic representation of the economy than model that assume a common import share. Indeed, the import share of export is generally smaller than for consumption.



\subsection{Domestic demand}



\noindent\textbf{Received margins on domestically produced commodity $m$} \\
\begin{dmath}
MGRD_{m} = \left( 1 - \varphi^{MGRM}_{m} \right) \; MGR_{m}
\end{dmath}

\noindent\textbf{Private final consumption of domestically produced commodity $c$} \\
\begin{dmath}
CHD_{c} = \left( 1 - \varphi^{CHM}_{c} \right) \; CH_{c}
\end{dmath}

\noindent\textbf{Public final consumption of domestically produced commodity $c$} \\
\begin{dmath}
GD_{c} = \left( 1 - \varphi^{GM}_{c} \right) \; G_{c}
\end{dmath}

\noindent\textbf{Margins received from imported commodity $m$} \\
\begin{dmath}
MGRM_{m} = \varphi^{MGRM}_{m} \; MGR_{m}
\end{dmath}

\noindent\textbf{Private final consumption of imported commodity $c$} \\
\begin{dmath}
CHM_{c} = \varphi^{CHM}_{c} \; CH_{c}
\end{dmath}

\noindent\textbf{Public final consumption of imported commodity $c$} \\
\begin{dmath}
GM_{c} = \varphi^{GM}_{c} \; G_{c}
\end{dmath}

\noindent\textbf{Import share of commodity $c$ on received margins} \\
\begin{dmath}
\varphi^{MGRM}_{m} = \frac{1}{\left( 1 + \frac{MGRD_{m}}{MGRM_{m}, t_{0}} \; \operatorname{exp} SUBST^{MGRM}_{m} \right)}
\end{dmath}

\noindent\textbf{Import share of commodity $c$ for household final consumption} \\
\begin{dmath}
\varphi^{CHM}_{c} = \frac{1}{\left( 1 + \frac{CHD_{c}}{CHM_{c}, t_{0}} \; \operatorname{exp} SUBST^{CHM}_{c} \right)}
\end{dmath}

\noindent\textbf{Import share $\varphi_c$ of commodity $c$ on the governement final consumption} \\
\begin{dmath}
\varphi^{GM}_{c} = \frac{1}{\left( 1 + \frac{GD_{c}}{GM_{c}, t_{0}} \; \operatorname{exp} SUBST^{GM}_{c} \right)}
\end{dmath}

\noindent\textbf{Notional substitution effect induced by a change in the relative price between imported and domestically produced commodity $c$ for margins received} \\
\begin{dmath}
\varDelta \left(SUBST^{n,MGRM}_{c}\right) = -\sigma^{MGRM}_{c} \; \varDelta \left(\operatorname{log} PMGRD_{c} - \operatorname{log} PMGRM_{c}\right)
\end{dmath}

\noindent\textbf{Notional substitution effect induced by a change in the relative price between imported and domestically produced commodity $c$ for households final consumption} \\
\begin{dmath}
\varDelta \left(SUBST^{n,CHM}_{c}\right) = -\sigma^{CHM}_{c} \; \varDelta \left(\operatorname{log} PCHD_{c} - \operatorname{log} PCHM_{c}\right)
\end{dmath}

\noindent\textbf{Notional substitution effect induced by a change in the relative price between imported and domestically produced commodity $c$ for government final consumption} \\
\begin{dmath}
\varDelta \left(SUBST^{n,GM}_{c}\right) = -\sigma^{GM}_{c} \; \varDelta \left(\operatorname{log} PGD_{c} - \operatorname{log} PGM_{c}\right)
\end{dmath}

\noindent\textbf{Intermediary consumption from sector $s$ in domestically produced commodity $c$} \\
\begin{dmath}
CID_{c, s} = \left( 1 - \varphi^{CIM}_{c, s} \right) \; CI_{c, s}
\end{dmath}

\noindent\textbf{Investment from sector $s$ in domestically produced commodity $c$} \\
\begin{dmath}
ID_{c, s} = \left( 1 - \varphi^{IM}_{c, s} \right) \; I_{c, s}
\end{dmath}

\noindent\textbf{Intermediary consumption from sector $s$ in imported commodity $c$} \\
\begin{dmath}
CIM_{c, s} = \varphi^{CIM}_{c, s} \; CI_{c, s}
\end{dmath}

\noindent\textbf{Investment from sector $s$ in imported commodity $c$} \\
\begin{dmath}
IM_{c, s} = \varphi^{IM}_{c, s} \; I_{c, s}
\end{dmath}

\noindent\textbf{Import share of intermediary consumption from sector $s$ in domestically produced commodity $c$} \\
\begin{dmath}
\varphi^{CIM}_{c, s} = \frac{1}{\left( 1 + \frac{CID_{c, s}}{CIM_{c, s}, t_{0}} \; \operatorname{exp} SUBST^{CIM}_{c, s} \right)}
\end{dmath}

\noindent\textbf{Import share of intermediary consumption from sector $s$ in imported commodity $c$} \\
\begin{dmath}
\varphi^{IM}_{c, s} = \frac{1}{\left( 1 + \frac{ID_{c, s}}{IM_{c, s}, t_{0}} \; \operatorname{exp} SUBST^{IM}_{c, s} \right)}
\end{dmath}

\noindent\textbf{Notional substitution effect induced by a change in the relative price between imported and domestic intermediary consumption in commodity $c$ from the sector $s$} \\
\begin{dmath}
\varDelta \left(SUBST^{n,CIM}_{c, s}\right) = -\sigma^{CIM}_{c, s} \; \varDelta \left(\operatorname{log} PCID_{c, s} - \operatorname{log} PCIM_{c, s}\right)
\end{dmath}

\noindent\textbf{Notional substitution effect induced by a change in the relative price between imported and domestic investment in commodity $c$ from the sector $s$} \\
\begin{dmath}
\varDelta \left(SUBST^{n,IM}_{c, s}\right) = -\sigma^{IM}_{c, s} \; \varDelta \left(\operatorname{log} PID_{c, s} - \operatorname{log} PIM_{c, s}\right)
\end{dmath}





\subsection{Exports}


\noindent\textbf{Exports of domestically produced commodity $c$} \\
\begin{dmath}
XD_{c} = \left( 1 - \varphi^{XM}_{c} \right) \; X_{c}
\end{dmath}

\noindent\textbf{Exports of imported commodity $c$} \\
\begin{dmath}
XM_{c} = \varphi^{XM}_{c} \; X_{c}
\end{dmath}

\noindent\textbf{Import share of commodity $c$ exports} \\
\begin{dmath}
\varphi^{XM}_{c} = \frac{1}{\left( 1 + \frac{XD_{c}}{XM_{c}, t_{0}} \; \operatorname{exp} SUBST^{XM}_{c} \right)}
\end{dmath}

\noindent\textbf{Notional substitution effect induced by a change in the relative price between imported and domestic products $c$ for exports} \\
\begin{dmath}
\varDelta \left(SUBST^{n,XM}_{c}\right) = -\sigma^{XM}_{c} \; \varDelta \left(\operatorname{log} PXD_{c} - \operatorname{log} PXM_{c}\right)
\end{dmath}

\noindent\textbf{Foreign demand for exports of commmodity $c$} \\
\begin{dmath}
\varDelta \left(\operatorname{log} X_{c}\right) = \varDelta \left(\operatorname{log} WD_{c}\right) + \varDelta \left(SUBST^{X}_{c}\right)
\end{dmath}

\noindent\textbf{Notional substitution effect induced by a change in the relative price between export prices and (converted in domestic currency) international prices for the commodity $c$} \\
\begin{dmath}
\varDelta \left(SUBST^{n,X}_{c}\right) = -\sigma^{X}_{c} \; \varDelta \left(\operatorname{log} PX_{c} - \operatorname{log} EXR . PWD_{c}\right)
\end{dmath}

\noindent\textbf{Balance of trade of commodity $c$} \\
\begin{dmath}
Bal^{Trade,VAL}_{c} = PX_{c} \; X_{c} - PM_{c} \; M_{c}
\end{dmath}

\noindent\textbf{Aggregate balance of trade} \\
\begin{dmath}
Bal^{Trade,VAL} = \sum_{c} Bal^{Trade,VAL}_{c}
\end{dmath}

\noindent\textbf{Balance of trade (in percent of GDP)} \\
\begin{dmath}
RBal^{Trade,VAL} = \frac{Bal^{Trade,VAL}}{\left( PGDP . GDP \right)}
\end{dmath}


\section{Demography}



\noindent\textbf{Working-age population} \\
\begin{dmath}
\varDelta \left(\operatorname{log} WAPop\right) = \varDelta \left(\operatorname{log} POP\right)
\end{dmath}
The working age population linearly grows with the total population. \\

\noindent\textbf{Labor force} \\
\begin{dmath}
LF = PARTR . WAPop
\end{dmath}
The Labor force depends on a participation rate of the working-age population. \\

\noindent\textbf{Labor force participation ratio} \\
\begin{dmath}
\varDelta \left(PARTR^{n}\right) = \varDelta \left(PARTR^{trend}\right) - \rho^{PART,UnR} . \varDelta \left(UnR\right)
\end{dmath}
Because of discouraged worker effect, the participation ratio depends generally negatively on the unemployment rate. \\

\noindent\textbf{Employment (ILO definition)} \\
\begin{dmath}
\varDelta \left(\operatorname{log} empl\right) = \varDelta \left(\operatorname{log} F_{L}\right)
\end{dmath}
In general, labor according to the national account differs from the employment according to the ILO definition. One reason is that labor is expressed in FTE (full time equivalent). To calculate the unemployment rate, one needs to use the employment according to the ILO definition. We assume that the average work duration is constant over time, implying stability of the employment to labor ratio. \\

\noindent\textbf{Unemployment} \\
\begin{dmath}
Un = LF - Empl
\end{dmath}
Unemployment is determined as the difference between the total active population with the one which is employed. \\

\noindent\textbf{Unemployment rate} \\
\begin{dmath}
UnR = \frac{Un}{LF}
\end{dmath}
The Unemployment rate is defined as the ratio between the total unemployment and the active population.


\section{Greenhouse gases emissions}


This file provides the equations defining the path of GreenHouse Gases (GHG) emissions. All emission types are expressed in CO2-equivalent to facilitate aggregation. For the same emission type (e.g. CO2), several equation are defined depending on the emission basis: intermediary consumption, household consumption or production. \\


\noindent\textbf{Emissions ghg related to the intermediary consumption of of commodity $c$ by sector $s$} \\
\begin{dmath}
\varDelta \left(\operatorname{log} EMS^{CI}_{ghg, c, s}\right) = \varDelta \left(\operatorname{log} CI_{c, s} \; IEMS^{CI}_{ghg, c, s}\right)
\end{dmath}
In practice only a few intermediaries generate emissions (e.g. coal, gas, petrol). IEMS_CI[ghg,c,s] is the corresponding emission intensity calibrated to 1 in the baseyear. It may change over time because of the increase of the share of biofuels. \\

\noindent\textbf{Emissions ghg related to the  materials consumption of sector $s$} \\
\begin{dmath}
\varDelta \left(\operatorname{log} EMS^{MAT}_{ghg, s}\right) = \varDelta \left(\operatorname{log} F_{MAT, s} \; IEMS^{MAT}_{ghg, s}\right)
\end{dmath}
This mainly corresponds to the CO2 emissions from decarbonation. \\

\noindent\textbf{Emissions ghg related to the final production of sector $s$} \\
\begin{dmath}
\varDelta \left(\operatorname{log} EMS^{Y}_{ghg, s}\right) = \varDelta \left(\operatorname{log} Y_{s} \; IEMS^{Y}_{ghg, s}\right)
\end{dmath}
This mainly correspond to the emissions from agriculture. \\

\noindent\textbf{Emissions ghg related to the household consumption $c$} \\
\begin{dmath}
\varDelta \left(\operatorname{log} EMS^{CH}_{ghg, c}\right) = \varDelta \left(\operatorname{log} CH_{c} \; IEMS^{CH}_{ghg, c}\right)
\end{dmath}

\noindent\textbf{Emissions ghg related to the intermediary consumption of commodity $c$} \\
\begin{dmath}
EMS^{CI}_{ghg, c} = \sum_{s} EMS^{CI}_{ghg, c, s}
\end{dmath}

\noindent\textbf{Emissions ghg related to the intermediary consumption by sector $s$} \\
\begin{dmath}
EMS^{CI}_{ghg, s} = \sum_{c} EMS^{CI}_{ghg, c, s}
\end{dmath}

\noindent\textbf{Emissions of the greehouse gas $ghg$ related to the intermediary consumption} \\
\begin{dmath}
EMS^{CI}_{ghg} = \sum_{s} EMS^{CI}_{ghg, s}
\end{dmath}
Aggregation by sector $s$. \\

\noindent\textbf{Emissions of the greehouse gas $ghg$ related to the intermediary consumption} \\
\begin{dmath}
EMS^{CI,bis}_{ghg} = \sum_{c} EMS^{CI}_{ghg, c}
\end{dmath}
Aggregation by commodity $c$. \\

\noindent\textbf{Emissions of the greehouse gas $ghg$ related to the total material consumption} \\
\begin{dmath}
EMS^{MAT}_{ghg} = \sum_{s} EMS^{MAT}_{ghg, s}
\end{dmath}

\noindent\textbf{Emissions of the greehouse gas $ghg$ related to the final production} \\
\begin{dmath}
EMS^{Y}_{ghg} = \sum_{s} EMS^{Y}_{ghg, s}
\end{dmath}

\noindent\textbf{Emissions of the greehouse gas $ghg$ related to the household final consumption} \\
\begin{dmath}
EMS^{CH}_{ghg} = \sum_{c} EMS^{CH}_{ghg, c}
\end{dmath}

\noindent\textbf{Aggregate emissions of the greehouse gas $ghg$} \\
\begin{dmath}
EMS_{ghg} = EMS^{CI}_{ghg} + EMS^{MAT}_{ghg} + EMS^{Y}_{ghg} + EMS^{CH}_{ghg}
\end{dmath}

\noindent\textbf{Aggregate emissions related to the intermediary consumption} \\
\begin{dmath}
EMS^{CI} = \sum_{ghg} EMS^{CI}_{ghg}
\end{dmath}

\noindent\textbf{Aggregate emissions related to the material consumption} \\
\begin{dmath}
EMS^{MAT} = \sum_{ghg} EMS^{MAT}_{ghg}
\end{dmath}

\noindent\textbf{Aggregate emissions related to the final production} \\
\begin{dmath}
EMS^{Y} = \sum_{ghg} EMS^{Y}_{ghg}
\end{dmath}

\noindent\textbf{Aggregate emissions related to the households final consumption} \\
\begin{dmath}
EMS^{CH} = \sum_{ghg} EMS^{CH}_{ghg}
\end{dmath}

\noindent\textbf{Aggregate emissions} \\
\begin{dmath}
EMS = EMS^{CI} + EMS^{MAT} + EMS^{Y} + EMS^{CH}
\end{dmath}

\noindent\textbf{Aggregate emissions by type of substance} \\
\begin{dmath}
EMS^{bis} = \sum_{ghg} EMS_{ghg}
\end{dmath}


\section{Other equations}





\subsection{Adjustment equations and anticipation}



\noindent\textbf{Mark-up in the sector $s$} \\
\begin{dmath}
\mu_{s} = \alpha^{\mu}_{s} \; \mu^{n}_{s} + \left( 1 - \alpha^{\mu}_{s} \right) \; \mu_{s, t-1}
\end{dmath}

\noindent\textbf{Expected inflation.} \\
\begin{dmath}
\varDelta \left(\operatorname{log} P^{e}\right) = \alpha^{Pe,P1} . \varDelta \left(\operatorname{log} P_{t-1}\right) + \left( 1 - \alpha^{Pe,P1} \right) . \varDelta \left(\operatorname{log} P^{e}_{t-1}\right)
\end{dmath}
This equation defines the expected inflation and not the expected price. $P^{e}$ does not necessary converge to $P$. If the wage equation is a WS curve, even in the very long term it may not converge. \\

\noindent\textbf{Expected production} \\
\begin{dmath}
\varDelta \left(\operatorname{log} Y^{e}_{s}\right) = \alpha^{Ye,Y}_{s} \; \varDelta \left(\operatorname{log} Y_{s}\right) + \left( 1 - \alpha^{Ye,Y}_{s} \right) \; \varDelta \left(\operatorname{log} Y^{e}_{s, t-1}\right)
\end{dmath}

\noindent\textbf{Quantity of Labor, Energy and Material inputs in sector $s$} \\
\begin{dmath}
\operatorname{log} F_{f, s} = \alpha^{{0},F}_{f, s} \; \operatorname{log} F^{n}_{f, s} + \left( 1 - \alpha^{{0},F}_{f, s} \right) \; \left( \operatorname{log} F_{f, s, t-1} + \varDelta \left(\operatorname{log} F^{e}_{f, s}\right) \right)
\end{dmath}

\noindent\textbf{Expected quantity of Labor, Energy and Material inputs in sector $s$} \\
\begin{dmath}
\varDelta \left(\operatorname{log} F^{e}_{f, s}\right) = \alpha^{{1},F}_{f, s} \; \varDelta \left(\operatorname{log} F^{e}_{f, s, t-1}\right) + \alpha^{{2},F}_{f, s} \; \varDelta \left(\operatorname{log} F_{f, s, t-1}\right) + \alpha^{{3},F}_{f, s} \; \varDelta \left(\operatorname{log} F^{n}_{f, s}\right)
\end{dmath}

\noindent\textbf{Capital stock of sector $s$} \\
\begin{dmath}
F_{K, s} = \left( 1 - \delta_{s} \right) \; F_{K, s, t-1} + IA_{s}
\end{dmath}

\noindent\textbf{Investment in sector $s$} \\


\begin{dmath}
\varDelta \left(\operatorname{log} IA_{s}\right) = \alpha^{IA,Ye}_{s} \; \varDelta \left(\operatorname{log} Y^{e}_{s}\right) + \alpha^{IA,IA1}_{s} \; \varDelta \left(\operatorname{log} IA_{s, t-1}\right) + \alpha^{IA,SUBST}_{s} \; \varDelta \left(SUBST^{F}_{K, s}\right) + \alpha^{IA,Kn}_{s} \; \left( \operatorname{log} F^{n}_{K, s, t-1} - \operatorname{log} F_{K, s, t-1} \right)
\end{dmath}









\noindent\textbf{Households final consumption of commodity $c$} \\
\begin{dmath}
\operatorname{log} CH_{c} = \alpha^{{0},CH}_{c} \; \operatorname{log} CH^{n}_{c} + \left( 1 - \alpha^{{0},CH}_{c} \right) \; \left( \operatorname{log} CH_{c, t-1} + \varDelta \left(\operatorname{log} CH^{e}_{c}\right) \right)
\end{dmath}

\noindent\textbf{Expected households final consumption of commodity $c$} \\
\begin{dmath}
\varDelta \left(\operatorname{log} CH^{e}_{c}\right) = \alpha^{{1},CH}_{c} \; \varDelta \left(\operatorname{log} CH^{e}_{c, t-1}\right) + \alpha^{{2},CH}_{c} \; \varDelta \left(\operatorname{log} CH_{c, t-1}\right) + \alpha^{{3},CH}_{c} \; \varDelta \left(\operatorname{log} CH^{n}_{c}\right)
\end{dmath}

\noindent\textbf{Production price of sector $s$} \\
\begin{dmath}
\operatorname{log} PY_{s} = \alpha^{{0},PY}_{s} \; \operatorname{log} PY^{n}_{s} + \left( 1 - \alpha^{{0},PY}_{s} \right) \; \left( \operatorname{log} PY_{s, t-1} + \varDelta \left(\operatorname{log} PY^{e}_{s}\right) \right)
\end{dmath}

\noindent\textbf{Expected production priceof sector $s$} \\
\begin{dmath}
\varDelta \left(\operatorname{log} PY^{e}_{s}\right) = \alpha^{{1},PY}_{s} \; \varDelta \left(\operatorname{log} PY^{e}_{s, t-1}\right) + \alpha^{{2},PY}_{s} \; \varDelta \left(\operatorname{log} PY_{s, t-1}\right) + \alpha^{{3},PY}_{s} \; \varDelta \left(\operatorname{log} PY^{n}_{s}\right)
\end{dmath}

\noindent\textbf{Wages of the sector $s$} \\
\begin{dmath}
\varDelta \left(\operatorname{log} W_{s}\right) = \alpha^{W,Wn}_{s} \; \varDelta \left(\operatorname{log} W^{n}_{s}\right) + \alpha^{W,W1}_{s} \; \varDelta \left(\operatorname{log} W_{s, t-1}\right) - \alpha^{W,W1Wn1}_{s} \; \operatorname{log} \frac{W_{s, t-1}}{W^{n}_{s, t-1}}
\end{dmath}

\noindent\textbf{Labor participation ratio} \\
\begin{dmath}
PARTR = \alpha^{{0},PARTR} . PARTR^{n} + \left( 1 - \alpha^{{0},PARTR} \right) . PARTR_{t-1}
\end{dmath}

\noindent\textbf{Interest rate} \\
\begin{dmath}
R = \alpha^{{0},R} . R^{n} + \left( 1 - \alpha^{{0},R} \right) . R_{t-1}
\end{dmath}

\noindent\textbf{ Households property income in value} \\
\begin{dmath}
\operatorname{log} PROP^{INC,H,VAL} = \alpha^{{0},PROP,INC,H,VAL} . \operatorname{log} PROP^{INC,H,VAL,n} + \left( 1 - \alpha^{{0},PROP,INC,H,VAL} \right) . \left( \operatorname{log} PROP^{INC,H,VAL}_{t-1} + \varDelta \left(\operatorname{log} PROP^{INC,H,VAL,e}\right) \right)
\end{dmath}

\noindent\textbf{Expected Households property income in value} \\
\begin{dmath}
\varDelta \left(\operatorname{log} PROP^{INC,H,VAL,e}\right) = \alpha^{{1},PROP,INC,H,VAL} . \varDelta \left(\operatorname{log} PROP^{INC,H,VAL,e}_{t-1}\right) + \alpha^{{2},PROP,INC,H,VAL} . \varDelta \left(\operatorname{log} PROP^{INC,H,VAL}_{t-1}\right) + \alpha^{{3},PROP,INC,H,VAL} . \varDelta \left(\operatorname{log} PROP^{INC,H,VAL,n}\right)
\end{dmath}

\noindent\textbf{Government property incomes in value} \\
\begin{dmath}
\operatorname{log} PROP^{INC,G,VAL} = \alpha^{{0},PROP,INC,G,VAL} . \operatorname{log} PROP^{INC,G,VAL,n} + \left( 1 - \alpha^{{0},PROP,INC,G,VAL} \right) . \left( \operatorname{log} PROP^{INC,G,VAL}_{t-1} + \varDelta \left(\operatorname{log} PROP^{INC,G,VAL,e}\right) \right)
\end{dmath}

\noindent\textbf{Expected Government property incomes in value} \\
\begin{dmath}
\varDelta \left(\operatorname{log} PROP^{INC,G,VAL,e}\right) = \alpha^{{1},PROP,INC,G,VAL} . \varDelta \left(\operatorname{log} PROP^{INC,G,VAL,e}_{t-1}\right) + \alpha^{{2},PROP,INC,G,VAL} . \varDelta \left(\operatorname{log} PROP^{INC,G,VAL}_{t-1}\right) + \alpha^{{3},PROP,INC,G,VAL} . \varDelta \left(\operatorname{log} PROP^{INC,G,VAL,n}\right)
\end{dmath}



\subsection{Substitutions}



\noindent\textbf{Substitution effect of the production factor $f$ in the sector $s$} \\
\begin{dmath}
SUBST^{F}_{f, s} = \alpha^{{6},F}_{f, s} \; SUBST^{n,F}_{f, s} + \left( 1 - \alpha^{{6},F}_{f, s} \right) \; SUBST^{F}_{f, s, t-1}
\end{dmath}

\noindent\textbf{Substitution effect of the domestic margin paid $m$ for the commodity $c$} \\
\begin{dmath}
SUBST^{MGPD}_{m, c} = \alpha^{{6},MGPD}_{m, c} \; SUBST^{n,MGPD}_{m, c} + \left( 1 - \alpha^{{6},MGPD}_{m, c} \right) \; SUBST^{MGPD}_{m, c, t-1}
\end{dmath}

\noindent\textbf{Substitution effect on the imported margin paid $m$ for the commodity $c$} \\
\begin{dmath}
SUBST^{MGPM}_{m, c} = \alpha^{{6},MGPM}_{m, c} \; SUBST^{n,MGPM}_{m, c} + \left( 1 - \alpha^{{6},MGPM}_{m, c} \right) \; SUBST^{MGPM}_{m, c, t-1}
\end{dmath}

\noindent\textbf{Substitution effect on the energy intermediate consumption $ce$ in the sector $s$} \\
\begin{dmath}
SUBST^{CI}_{ce, s} = \alpha^{{6},CI}_{ce, s} \; SUBST^{n,CI}_{ce, s} + \left( 1 - \alpha^{{6},CI}_{ce, s} \right) \; SUBST^{CI}_{ce, s, t-1}
\end{dmath}

\noindent\textbf{Substitution effect on the transportation intermediate consumption $ce$ in the sector $s$} \\
\begin{dmath}
SUBST^{CI}_{ct, s} = \alpha^{{6},CI}_{ct, s} \; SUBST^{n,CI}_{ct, s} + \left( 1 - \alpha^{{6},CI}_{ct, s} \right) \; SUBST^{CI}_{ct, s, t-1}
\end{dmath}

\noindent\textbf{Substitution effect on the imported margin received for the commodity $m$} \\
\begin{dmath}
SUBST^{MGRM}_{m} = \alpha^{{6},MGRM}_{m} \; SUBST^{n,MGRM}_{m} + \left( 1 - \alpha^{{6},MGRM}_{m} \right) \; SUBST^{MGRM}_{m, t-1}
\end{dmath}

\noindent\textbf{Substitution effect on the imported households final consumption for the commodity $c$} \\
\begin{dmath}
SUBST^{CHM}_{c} = \alpha^{{6},CHM}_{c} \; SUBST^{n,CHM}_{c} + \left( 1 - \alpha^{{6},CHM}_{c} \right) \; SUBST^{CHM}_{c, t-1}
\end{dmath}

\noindent\textbf{Substitution effect on the imported government final consumption for the commodity $c$} \\
\begin{dmath}
SUBST^{GM}_{c} = \alpha^{{6},GM}_{c} \; SUBST^{n,GM}_{c} + \left( 1 - \alpha^{{6},GM}_{c} \right) \; SUBST^{GM}_{c, t-1}
\end{dmath}

\noindent\textbf{Substitution effect on the government final consumption for the imported commodity $c$} \\
\begin{dmath}
SUBST^{XM}_{c} = \alpha^{{6},XM}_{c} \; SUBST^{n,XM}_{c} + \left( 1 - \alpha^{{6},XM}_{c} \right) \; SUBST^{XM}_{c, t-1}
\end{dmath}

\noindent\textbf{Substitution effect on the intermediate consumption for the imported commodity $c$ in the sector $s$} \\
\begin{dmath}
SUBST^{CIM}_{c, s} = \alpha^{{6},CIM}_{c, s} \; SUBST^{n,CIM}_{c, s} + \left( 1 - \alpha^{{6},CIM}_{c, s} \right) \; SUBST^{CIM}_{c, s, t-1}
\end{dmath}

\noindent\textbf{Substitution effect on the investment for the imported commodity $c$ in the sector $s$} \\
\begin{dmath}
SUBST^{IM}_{c, s} = \alpha^{{6},IM}_{c, s} \; SUBST^{n,IM}_{c, s} + \left( 1 - \alpha^{{6},IM}_{c, s} \right) \; SUBST^{IM}_{c, s, t-1}
\end{dmath}

\noindent\textbf{Substitution effect on the exports of the commodity $c$} \\
\begin{dmath}
SUBST^{X}_{c} = \alpha^{{6},X}_{c} \; SUBST^{n,X}_{c} + \left( 1 - \alpha^{{6},X}_{c} \right) \; SUBST^{X}_{c, t-1}
\end{dmath}
\newpage
        \section{Glossary}
        \normalsize
        \begin{longtable}{@{}p{4cm}p{9cm}@{}} 
$Bal^{G,Prim,VAL}$ & Primary balance of the Government in value (deficit). \\
 \midrule 
$Bal^{G,Prim,VAL,bis}$ & Primary balance of the Government in value (deficit)from an alternative expression \\
 \midrule 
$Bal^{G,Tot,VAL}$ & Total balance of the Government in value (deficit) \\
 \midrule 
$Bal^{Trae,VAL}$ & Aggregate balance of trade \\
 \midrule 
$Bal^{Trae,VAL}_{c}$ & Balance of trade of commodity $c$ \\
 \midrule 
$C_{E, s}$ & Energy costs in sector $s$ \\
 \midrule 
$C_{E}$ & Aggregate cost of energy \\
 \midrule 
$C_{K, s}$ & Capital cost in sector $s$ \\
 \midrule 
$C_{K}$ & Aggregate cost of capital \\
 \midrule 
$C_{L, s}$ & Labor cost in sector $s$ \\
 \midrule 
$C_{L}$ & Aggregate cost of labor \\
 \midrule 
$C_{MAT, s}$ & Materials costs in sector $s$ \\
 \midrule 
$C_{MAT}$ & Aggregate cost of materials \\
 \midrule 
$CH$ & Aggregate household final consumption, expressed at market price \\
 \midrule 
$CH_{c}$ & Households final consumption of commodity $c$ \\
 \midrule 
$CH^{e}_{c}$ & Expected households final consumption of commodity $c$ \\
 \midrule 
$CH^{n}_{c}$ & Households  final consumption of commodity $c$ \\
 \midrule 
$CH^{n,VAL}$ & Aggregate notional households final consumption in value \\
 \midrule 
$CHD$ & Aggregate domestically produced final consumption, expressed at market price \\
 \midrule 
$CHD_{c}$ & Private final consumption of domestically produced commodity $c$ \\
 \midrule 
$CHM$ & Aggregate imported households final consumption, expressed at market price \\
 \midrule 
$CHM_{c}$ & Private final consumption of imported commodity $c$ \\
 \midrule 
$CI$ & Aggregate intermediate consumption, expressed at market price \\
 \midrule 
$CI_{c}$ & Intermediate consumption of commodity $c$, expressed at market price \\
 \midrule 
$CI_{ce, s}$ & Energy input demand by type of energy $ce$ by sector $s$ \\
 \midrule 
$CI_{cmo, s}$ & Demand for material commodity $cmo$ by sector $s$ \\
 \midrule 
$CI_{ct, s}$ & Demand for transport commodity $ct$ by sector $s$ \\
 \midrule 
$CI_{s}$ & Intermediate consumption of sector $s$, expressed at market price \\
 \midrule 
$CI^{bis}$ & Intermediate consumption of sector $s$, expressed at market price (for verification) \\
 \midrule 
$CID$ & Aggregate domestically produced intermediate consumption, expressed at market price \\
 \midrule 
$CID_{c, s}$ & Intermediary consumption from sector $s$ in domestically produced commodity $c$ \\
 \midrule 
$CID_{c}$ & Quantity of domestically produced commodity $c$ used as intermediary consumption, expressed at market price \\
 \midrule 
$CID_{s}$ & Domestically produced intermediate consumption of sector $s$, expressed at basic price, expressed at market price \\
 \midrule 
$CIM$ &  Aggregate imported intermediate consumption, expressed at market price \\
 \midrule 
$CIM_{c, s}$ & Intermediary consumption from sector $s$ in imported commodity $c$ \\
 \midrule 
$CIM_{c}$ & Quantity of imported commodity $c$ used as intermediary consumption, expressed at market price \\
 \midrule 
$CIM_{s}$ & Imported intermediate consumption of sector $s$, expressed at market price \\
 \midrule 
$CU_{s}$ & Unit cost of production in sector $s$ \\
 \midrule 
$CU^{n}_{s}$ & Notional unit cost of production in sector $s$ \\
 \midrule 
$CUR_{s}$ & Capacity Utilisation ratio of the sector $s$ \\
 \midrule 
$DEBT^{G,VAL}$ & Government's debt in value \\
 \midrule 
$DISPINC^{AT,VAL}$ & Disposable income after tax in value \\
 \midrule 
$DISPINC^{BT,VAL}$ & Disposable income before tax in value \\
 \midrule 
$DS$ & Aggregate change in inventories, expressed at market price \\
 \midrule 
$DS_{c}$ & Change in inventories of commodity $c$, expressed at market price \\
 \midrule 
$DSD$ & Aggregate domestically produced change in inventories, expressed at market price \\
 \midrule 
$DSM$ & Aggregate imported change in inventories, expressed at market price \\
 \midrule 
$empl$ & Employment (ILO definition) \\
 \midrule 
$EMS$ & Aggregate emissions \\
 \midrule 
$EMS_{ghg}$ & Aggregate emissions of the greehouse gas $ghg$ \\
 \midrule 
$EMS^{CH}$ & Aggregate emissions related to the households final consumption \\
 \midrule 
$EMS^{CH}_{ghg, c}$ & Emissions ghg related to the household consumption $c$ \\
 \midrule 
$EMS^{CH}_{ghg}$ & Emissions of the greehouse gas $ghg$ related to the household final consumption \\
 \midrule 
$EMS^{CI}$ & Aggregate emissions related to the intermediary consumption \\
 \midrule 
$EMS^{CI}_{ghg, c, s}$ & Emissions ghg related to the intermediary consumption of of commodity $c$ by sector $s$ \\
 \midrule 
$EMS^{CI}_{ghg, c}$ & Emissions ghg related to the intermediary consumption of commodity $c$ \\
 \midrule 
$EMS^{CI}_{ghg, s}$ & Emissions ghg related to the intermediary consumption by sector $s$ \\
 \midrule 
$EMS^{CI}_{ghg}$ & Emissions of the greehouse gas $ghg$ related to the intermediary consumption \\
 \midrule 
$EMS^{CI,bis}_{ghg}$ & Emissions of the greehouse gas $ghg$ related to the intermediary consumption \\
 \midrule 
$EMS^{MAT}$ & Aggregate emissions related to the material consumption \\
 \midrule 
$EMS^{MAT}_{ghg, s}$ & Emissions ghg related to the  materials consumption of sector $s$ \\
 \midrule 
$EMS^{MAT}_{ghg}$ & Emissions of the greehouse gas $ghg$ related to the total material consumption \\
 \midrule 
$EMS^{Y}$ & Aggregate emissions related to the final production \\
 \midrule 
$EMS^{Y}_{ghg, s}$ & Emissions ghg related to the final production of sector $s$ \\
 \midrule 
$EMS^{Y}_{ghg}$ & Emissions of the greehouse gas $ghg$ related to the final production \\
 \midrule 
$EMS^{bis}$ & Aggregate emissions by type of substance \\
 \midrule 
$F_{E}$ & Aggregate energy input \\
 \midrule 
$F_{f, s}$ & Quantity of Labor, Energy and Material inputs in sector $s$ \\
 \midrule 
$F_{K, s}$ & Capital stock of sector $s$ \\
 \midrule 
$F_{K}$ & Aggregate capital input \\
 \midrule 
$F_{L}$ & Aggregate labor input \\
 \midrule 
$F_{MAT}$ & Aggregate materials input \\
 \midrule 
$F^{e}_{f, s}$ & Expected quantity of Labor, Energy and Material inputs in sector $s$ \\
 \midrule 
$F^{n}_{f, s}$ & Demand for production factor $f$ of sector $s$ \\
 \midrule 
$G$ & Aggregate Government final consumption, expressed at market price \\
 \midrule 
$G_{c}$ &  Government final consumption expenditure of commodity $c$ \\
 \midrule 
$GD$ & Aggregate domestically produced Government final consumption, expressed at market price \\
 \midrule 
$GD_{c}$ & Public final consumption of domestically produced commodity $c$ \\
 \midrule 
$GDP$ & GDP (expenditure definition) \\
 \midrule 
$GDP_{c}$ & GDP of commodity $c$ (expenditure definition) \\
 \midrule 
$GDP4$ &   GDP (income definition) \\
 \midrule 
$GDP^{bis}$ & GDP (expenditure definition, for verification) \\
 \midrule 
$GDP^{ter}$ &  GDP (production definition) \\
 \midrule 
$GM$ & Aggregate imported Government final consumption, expressed at market price \\
 \midrule 
$GM_{c}$ & Public final consumption of imported commodity $c$ \\
 \midrule 
$GOS$ & Aggregate gross operating surplus \\
 \midrule 
$GOS_{s}$ & Gross operating surplus of sector $s$ \\
 \midrule 
$GOS^{VAL}_{s}$ & Gross operating surplus of sector $s$ expressed in value \\
 \midrule 
$GR^{PROG}_{E, s}$ & Endogenous energy efficiency \\
 \midrule 
$I$ & Aggregate investment, expressed at market price \\
 \midrule 
$I_{c, s}$ & Investment use of commodity $c$ by sector $s$ \\
 \midrule 
$I_{c}$ & Investment in commodity $c$, expressed at market price \\
 \midrule 
$I_{s}$ & Investment of sector $s$, expressed at market price \\
 \midrule 
$IA_{s}$ & Investment in sector $s$ \\
 \midrule 
$I^{bis}$ & Investment of sector $s$, expressed at market price (for verification) \\
 \midrule 
$ID$ & Aggregate domestically produced investment, expressed at market price \\
 \midrule 
$ID_{c, s}$ & Investment from sector $s$ in domestically produced commodity $c$ \\
 \midrule 
$ID_{c}$ & Quantity of imported commodity $c$ used as investment, expressed at market price \\
 \midrule 
$ID_{s}$ & Domestically produced investment of sector $s$, expressed at market price \\
 \midrule 
$IM$ & Aggregate imported investment, expressed at market price \\
 \midrule 
$IM_{c, s}$ & Investment from sector $s$ in imported commodity $c$ \\
 \midrule 
$IM_{c}$ & Quantity of imported commodity $c$ used as investment, expressed at market price \\
 \midrule 
$IM_{s}$ & Imported investment of sector $s$, expressed at market price \\
 \midrule 
$INC^{G,VAL}$ & Incomes of the Government in value \\
 \midrule 
$INC^{SOC,TAX,VAL}$ & Income \& Social Taxes in value \\
 \midrule 
$LF$ & Labor force \\
 \midrule 
$M$ & Imports, expressed at basic price \\
 \midrule 
$M_{c}$ & Imports of commodity $c$, expressed at basic price \\
 \midrule 
$\mu_{c}$ & Average mark-up on commodity $c$ \\
 \midrule 
$\mu_{s}$ & Mark-up in the sector $s$ \\
 \midrule 
$\mu^{n}_{s}$ & Notional mark-up of the sector $s$ \\
 \midrule 
$\mu^{n2}_{s}$ & Notional mark-up of the sector $s$ (definition 2) \\
 \midrule 
$M^{bis}_{c}$ & Imports of commodity $c$, expressed at basic price (for verification) \\
 \midrule 
$MGP_{m, c}$ & Margins paid to commodity $m$ on commodity $c$ \\
 \midrule 
$MGPD$ & Margins paid on domestically produced commodities \\
 \midrule 
$MGPD_{c}$ & Margins paid on the domestically produced commodity $c$ \\
 \midrule 
$MGPD_{m, c}$ & Margins paid to commodity $m$ on the domestic commodity $c$ \\
 \midrule 
$MGPM$ & Margins paid on imported commodities \\
 \midrule 
$MGPM_{c}$ & Margins paid on imported commodity $c$ \\
 \midrule 
$MGPM_{m, c}$ & Margins paid to commodity $m$ on the imported commodity $c$ \\
 \midrule 
$MGR$ & Aggregate recieved margins \\
 \midrule 
$MGR_{m}$ & Margins received by commodity $m$, expressed at market price \\
 \midrule 
$MGR^{bis}_{c}$ & Margins received by commodity $m$, expressed at market price (for verification) \\
 \midrule 
$MGRD$ & Aggregate margins received on domestically produced commodities, expressed at market price \\
 \midrule 
$MGRD_{m}$ & Received margins on domestically produced commodity $m$ \\
 \midrule 
$MGRM$ & Aggregate margins received on imported commodities, expressed at market price \\
 \midrule 
$MGRM_{m}$ & Margins received from imported commodity $m$ \\
 \midrule 
$MPS^{n}$ & Notional marginal propensity to save \\
 \midrule 
$MS_{c}$ & Quantity of imported commodity $c$ expressed at selling price \\
 \midrule 
$NCH$ & Necessary households final consumption of commodity $c$ \\
 \midrule 
$NOS$ & Aggregate net operating surplus \\
 \midrule 
$NOS_{s}$ & Net operating surplus of sector $s$ \\
 \midrule 
$NOS^{VAL}_{s}$ & Net operating surplus of sector $s$ expressed in value \\
 \midrule 
$NTAXI$ & Net taxes on production in volume \\
 \midrule 
$NTAXI_{s}$ & Net taxes on production of sector $s$ in volume \\
 \midrule 
$NTAXI^{VAL}$ & Net taxes on production in value \\
 \midrule 
$NTAXI^{VAL}_{s}$ & Net taxes on production of sector $s$ in value \\
 \midrule 
$NTAXP$ & Aggregate net taxes on commodity $c$ in volume \\
 \midrule 
$NTAXP_{c}$ & Net taxes on commodity $c$ in volume \\
 \midrule 
$NTAXP^{VAL}_{c}$ & Net taxes on commodity $c$ in value \\
 \midrule 
$NTAXPD_{c}$ & Net taxes on domestically produced commodity $c$ in volume \\
 \midrule 
$NTAXPD^{VAL}_{c}$ & Net taxes on domestically produced commodity $c$ in value \\
 \midrule 
$NTAXPM_{c}$ & Net taxes on imported commodity $c$ in volume \\
 \midrule 
$NTAXPM^{VAL}_{c}$ & Net taxes on imported commodity $c$ in value \\
 \midrule 
$P$ & Consumer Price Index \\
 \midrule 
$P^{e}$ & Expected inflation. \\
 \midrule 
$PARTR$ & Labor participation ratio \\
 \midrule 
$PARTR^{n}$ & Labor force participation ratio \\
 \midrule 
$PCH$ & Aggregate market price for household final (consumer price index) \\
 \midrule 
$PCH_{c}$ & Price of commodity $c$ for household final consumption expenditure \\
 \midrule 
$PCH^{CES}$ & Consumption price \\
 \midrule 
$PCHD$ & Aggregate market price for domestically produced households final consumption \\
 \midrule 
$PCHD_{c}$ & Price of domestically produced commodity $c$ for households final consumption expenditure \\
 \midrule 
$PCHM$ & Aggregate market price for imported households final consumption \\
 \midrule 
$PCHM_{c}$ & Price of imported commodity $c$ for households final consumption expenditure \\
 \midrule 
$PCI$ & Aggregate market price for intermediate consumption \\
 \midrule 
$PCI_{c, s}$ & Price of commodity $c$ for sector $s$ for intermediary consumption use \\
 \midrule 
$PCI_{c}$ & Market price of the intermediate consumption of commodity $c$ \\
 \midrule 
$PCI_{s}$ & Market price of intermediate consumption of sector $s$ \\
 \midrule 
$PCI^{bis}$ & Market price of intermediate consumption of sector $s$ (for verification) \\
 \midrule 
$PCID$ & Aggregate market price for domestically produced intermediate consumption \\
 \midrule 
$PCID_{c, s}$ & Price of domestically produced commodity $c$ for sector $s$ for intermediate consumption use \\
 \midrule 
$PCID_{c}$ & Market price for the domestically produced commodity $c$ used as intermediary consumption \\
 \midrule 
$PCID_{s}$ & Market price of domestically produced intermediate consumption of sector $s$ \\
 \midrule 
$PCIM$ & Aggregate market price for imported intermediate consumption \\
 \midrule 
$PCIM_{c, s}$ & Price of imported commodity $c$ for sector $s$ for intermediate consumption use \\
 \midrule 
$PCIM_{c}$ & Market price for imported commodity $c$ used as intermediary consumption \\
 \midrule 
$PCIM_{s}$ & Market price of imported intermediate consumption of sector $s$ \\
 \midrule 
$PDS$ & Aggregate market price for change in inventories \\
 \midrule 
$PDS_{c}$ & Market price of the change in inventories of commodity $c$ \\
 \midrule 
$PDSD$ & Aggregate market price for domestically produced change in inventories \\
 \midrule 
$PDSD_{c}$ & Price of domestically produced commodity $c$ for change in inventories use \\
 \midrule 
$PDSM$ & Aggregate market price for imported change in inventories \\
 \midrule 
$PDSM_{c}$ & Price of imported commodity $c$ for change in inventories use \\
 \midrule 
$PE_{s}$ & Energy price for sector $s$ \\
 \midrule 
$PG$ & Aggregate market price for Government final consumption \\
 \midrule 
$PG_{c}$ & Price of commodity $c$ for government final consumption expenditure \\
 \midrule 
$PGD$ & Aggregate market price for domestically produced Government final consumption \\
 \midrule 
$PGD_{c}$ & Price of domestically produced commodity $c$ for government final consumption expenditure \\
 \midrule 
$PGDP$ & Price of GDP (expenditure definition) \\
 \midrule 
$PGDP_{c}$ & Price of GDP of commodity $c$ (expenditure definition) \\
 \midrule 
$PGDP4$ & Price of GDP (income definition) \\
 \midrule 
$PGDP^{bis}$ & Price of GDP (expenditure definition, for verification) \\
 \midrule 
$PGDP^{ter}$ & Price of GDP (production definition) \\
 \midrule 
$PGM$ & Aggregate market price for imported Government final consumption \\
 \midrule 
$PGM_{c}$ & Price of imported commodity $c$ for government final consumption expenditure \\
 \midrule 
$PGOS$ & Price of the aggregate gross operating surplus \\
 \midrule 
$\varphi_{E, ce, s}$ & Share of energy input $ce$  on total energy use by sector $s$ \\
 \midrule 
$\varphi_{f, s}$ & Share of production factor $f$ of sector $s$ \\
 \midrule 
$\varphi^{CH}_{c}$ & Share of commodity $c$ in the household consumption \\
 \midrule 
$\varphi^{CHM}_{c}$ & Import share of commodity $c$ for household final consumption \\
 \midrule 
$\varphi^{CIM}_{c, s}$ & Import share of intermediary consumption from sector $s$ in domestically produced commodity $c$ \\
 \midrule 
$\varphi^{GM}_{c}$ & Import share $\varphi_c$ of commodity $c$ on the governement final consumption \\
 \midrule 
$\varphi^{IM}_{c, s}$ & Import share of intermediary consumption from sector $s$ in imported commodity $c$ \\
 \midrule 
$\varphi^{MCH}_{c}$ & Share of commodity $c$ in the marginal household consumption \\
 \midrule 
$\varphi^{MGPD}_{m, c}$ & Market share of the margin-making sector $m$  for the commodity $c$ \\
 \midrule 
$\varphi^{MGPM}_{m, c}$ &  share of the margin type $m$ on total margins paid on the domestic commodity $c$ \\
 \midrule 
$\varphi^{MGRM}_{m}$ & Import share of commodity $c$ on received margins \\
 \midrule 
$\varphi^{TRSP}_{ct, s}$ & Share for transport $ct$ use in total transport by sector $s$ \\
 \midrule 
$\varphi^{XM}_{c}$ & Import share of commodity $c$ exports \\
 \midrule 
$PI$ & Aggregate market price for investment \\
 \midrule 
$PI_{c}$ & Market price of the investment in commodity $c$ \\
 \midrule 
$PI_{s}$ & Market price of investment of sector $s$ \\
 \midrule 
$PI^{bis}$ & Market price of investment of sector $s$ (for verification) \\
 \midrule 
$PID$ & Aggregate market price for domestically produced investment \\
 \midrule 
$PID_{c, s}$ & Price of domestically produced commodity $c$ for investment use \\
 \midrule 
$PID_{c}$ & Market price for domestically produced commodity $c$ used as investment \\
 \midrule 
$PID_{s}$ & Market price of domestically produced investment of sector $s$ \\
 \midrule 
$PIM$ & Aggregate market price for imported investment \\
 \midrule 
$PIM_{c, s}$ & Price of imported commodity $c$ for investment use \\
 \midrule 
$PIM_{c}$ & Market price for imported commodity $c$ used as investment \\
 \midrule 
$PIM_{s}$ & Market price of imported investment of sector $s$ \\
 \midrule 
$PK_{s}$ & Price of capital in sector $s$ \\
 \midrule 
$PM$ & Aggregate basic price of imports \\
 \midrule 
$PM_{c}$ & Price of imported commodity $c$ \\
 \midrule 
$PMAT_{s}$ & Materials price for sector $s$ \\
 \midrule 
$PM^{bis}_{c}$ & Basic price of imports of commodity $c$ (for verification) \\
 \midrule 
$PMGP_{m, c}$ & Price of the margins paid to commodity $m$ on commodity $c$ \\
 \midrule 
$PMGPD$ & Aggregate price of the margins paid on domestically produced commodity \\
 \midrule 
$PMGPD_{c}$ & Price of the margins paid on domestically produced commodity $c$ \\
 \midrule 
$PMGPD_{m, c}$ & Price of the margins paid to commodity $m$ on domestically produced commodity $c$ \\
 \midrule 
$PMGPM$ & Aggregate price of the margins paid on imported commodities \\
 \midrule 
$PMGPM_{c}$ & Price of the margins paid on imported commodity $c$ \\
 \midrule 
$PMGPM_{m, c}$ & Price of the margins paid to commodity $m$ on imported commodity $c$ \\
 \midrule 
$PMGR$ & Aggregate market price for recieved margins \\
 \midrule 
$PMGR_{m}$ & Market price of the margins received by commodity $m$ \\
 \midrule 
$PMGR^{bis}_{c}$ & Market price of the margins received by commodity $m$ (for verification) \\
 \midrule 
$PMGRD$ & Aggregate market price for the margins received on domestically produced commodities \\
 \midrule 
$PMGRD_{c}$ & Price of margins received on domestically produced commodity $c$ \\
 \midrule 
$PMGRM$ & Aggregate market price for the margins received on imported commodities \\
 \midrule 
$PMGRM_{c}$ & Price of margins received on imported commodity $c$ \\
 \midrule 
$PMS_{c}$ & Selling price for imported commodity $c$ \\
 \midrule 
$PNCH$ & Price of necessary households consumption of commodity $c$ \\
 \midrule 
$PNOS$ & Price of the aggregate net operating surplus \\
 \midrule 
$PNTAXP$ & Aggregate net taxes on commodity $c$ in value \\
 \midrule 
$PQ$ & Aggregate market price for production \\
 \midrule 
$PQ_{c}$ & Market price of the production of commodity $c$ \\
 \midrule 
$PQD$ & Aggregate market price for domestically produced commodities \\
 \midrule 
$PQD_{c}$ & Market price for the domestically produced commodity $c$ \\
 \midrule 
$PQM$ & Aggregate market price for imported commodities \\
 \midrule 
$PQM_{c}$ & Market price for imported commodity $c$ \\
 \midrule 
$PROG_{f, s}$ & Technical progress of the production factor $f$ in the sector $s$ \\
 \midrule 
$PROP^{INC,G,VAL}$ & Government property incomes in value \\
 \midrule 
$PROP^{INC,G,VAL,e}$ & Expected Government property incomes in value \\
 \midrule 
$PROP^{INC,G,VAL,n}$ & Notional Property incomes of the Government in value \\
 \midrule 
$PROP^{INC,H,VAL}$ &  Households property income in value \\
 \midrule 
$PROP^{INC,H,VAL,e}$ & Expected Households property income in value \\
 \midrule 
$PROP^{INC,H,VAL,n}$ & Property incomes in value \\
 \midrule 
$PRSSC$ & Total employers' social security contribution expressed in consumer price \\
 \midrule 
$PRSSC_{s}$ & Price of RSSC for sector $s$ \\
 \midrule 
$PVA$ & Value-added price \\
 \midrule 
$PWAGES$ & Gross wage index paid by sectors \\
 \midrule 
$PWAGES_{s}$ & Price Index for gross wages \\
 \midrule 
$PX$ & Aggregate market price for exports \\
 \midrule 
$PX_{c}$ & Price of commodity $c$ for exports use \\
 \midrule 
$PXD$ & Aggregate market price for domestically produced exports \\
 \midrule 
$PXD_{c}$ & Price of domestically produced commodity $c$ for export use \\
 \midrule 
$PXM$ & Aggregate market price for imported exports (re-exports) \\
 \midrule 
$PXM_{c}$ & Price of imported commodity $c$ for export use \\
 \midrule 
$PY$ & Basic price of aggregate production \\
 \midrule 
$PY_{s}$ & Production price of sector $s$ \\
 \midrule 
$PY^{e}_{s}$ & Expected production priceof sector $s$ \\
 \midrule 
$PY^{n}_{s}$ & Notional production price of sector $s$ \\
 \midrule 
$PYQ$ & Aggregate basic price of domestic production \\
 \midrule 
$PYQ_{c}$ & Domestic production price of commodity $c$ \\
 \midrule 
$PYQ^{bis}_{c}$ & Basic price of the production of commodity $c$ (for verification) \\
 \midrule 
$PYQS_{c}$ & Selling price of commodity $c$ \\
 \midrule 
$Q$ & Aggregate production, expressed at market price \\
 \midrule 
$Q_{c}$ & Production of commodity $c$, expressed at market price \\
 \midrule 
$QD$ & Aggregate domestically produced commodities, expressed at market price \\
 \midrule 
$QD_{c}$ & Quantity of domestically produced commodity $c$ expressed at market price \\
 \midrule 
$QM$ & Aggregate imported commodities, expressed at market price \\
 \midrule 
$QM_{c}$ & Quantity of imported commodity $c$ expressed at market price \\
 \midrule 
$R$ & Interest rate \\
 \midrule 
$R_{s}$ & Interest rate paid on capital by sector $s$ \\
 \midrule 
$r^{DEBT,G}$ & Interest rate paid by the Governement on its debt \\
 \midrule 
$R^{n}$ & Notional interest rate of the Central Bank (Taylor rule) \\
 \midrule 
$RBal^{G,Prim,VAL}$ & Primary balance of the Government in value (in percent of GDP) \\
 \midrule 
$RBal^{G,Tot,VAL}$ & Total balance of the Government in value (in percent of GDP) \\
 \midrule 
$RBal^{Trae,VAL}$ & Balance of trade (in percent of GDP) \\
 \midrule 
$RDEBT^{G,VAL}$ & Ratio of the Government's debt in value (in percent of GDP) \\
 \midrule 
$RRSSC$ & Average employers' social security contribution rate \\
 \midrule 
$RSAV^{G,VAL}$ & Government's savings rate in value (in percent of GDP) \\
 \midrule 
$RSAV^{H,VAL}$ & Households savings rate \\
 \midrule 
$RSSC$ & Price of RSSC \\
 \midrule 
$RSSC_{s}$ & Employers' social security contribution paid by sector $s$ expressed in consumer price \\
 \midrule 
$SAV^{G,VAL}$ & Savings of the Government in value (Net lending/borrowing: published deficit/savings of the Government) \\
 \midrule 
$SAV^{H,VAL}$ & Households savings in value \\
 \midrule 
$SOC^{BENF,VAL}$ & Social benefits in value \\
 \midrule 
$SPEND^{G,VAL}$ & Spendings of the Government in value \\
 \midrule 
$Stock^{SAV,H,VAL}$ & Households savings stock \\
 \midrule 
$SUBST^{CHM}_{c}$ & Substitution effect on the imported households final consumption for the commodity $c$ \\
 \midrule 
$SUBST^{CI}_{ce, s}$ & Substitution effect on the energy intermediate consumption $ce$ in the sector $s$ \\
 \midrule 
$SUBST^{CI}_{ct, s}$ & Substitution effect on the transportation intermediate consumption $ce$ in the sector $s$ \\
 \midrule 
$SUBST^{CIM}_{c, s}$ & Substitution effect on the intermediate consumption for the imported commodity $c$ in the sector $s$ \\
 \midrule 
$SUBST^{F}_{f, s}$ & Substitution effect of the production factor $f$ in the sector $s$ \\
 \midrule 
$SUBST^{GM}_{c}$ & Substitution effect on the imported government final consumption for the commodity $c$ \\
 \midrule 
$SUBST^{IM}_{c, s}$ & Substitution effect on the investment for the imported commodity $c$ in the sector $s$ \\
 \midrule 
$SUBST^{MGPD}_{m, c}$ & Substitution effect of the domestic margin paid $m$ for the commodity $c$ \\
 \midrule 
$SUBST^{MGPM}_{m, c}$ & Substitution effect on the imported margin paid $m$ for the commodity $c$ \\
 \midrule 
$SUBST^{MGRM}_{m}$ & Substitution effect on the imported margin received for the commodity $m$ \\
 \midrule 
$SUBST^{n,CHM}_{c}$ & Notional substitution effect induced by a change in the relative price between imported and domestically produced commodity $c$ for households final consumption \\
 \midrule 
$SUBST^{n,CI}_{ce, s}$ & Notional substitution effect between the energy commodity $ce$ and the over energy commodities $cee$ for the sector $s$ \\
 \midrule 
$SUBST^{n,CI}_{ct, s}$ & Notional substitution effect between the transport $ct$ and the over transports $mt$ for the sector $s$ \\
 \midrule 
$SUBST^{n,CIM}_{c, s}$ & Notional substitution effect induced by a change in the relative price between imported and domestic intermediary consumption in commodity $c$ from the sector $s$ \\
 \midrule 
$SUBST^{n,F}_{f, s}$ & Notional substitution effect between the input $f$ and the over inputs $ff$ \\
 \midrule 
$SUBST^{n,GM}_{c}$ & Notional substitution effect induced by a change in the relative price between imported and domestically produced commodity $c$ for government final consumption \\
 \midrule 
$SUBST^{n,IM}_{c, s}$ & Notional substitution effect induced by a change in the relative price between imported and domestic investment in commodity $c$ from the sector $s$ \\
 \midrule 
$SUBST^{n,MGPD}_{m, c}$ & Notional substitution between margin-making sectors $m$ for the domestically produced commodity $c$ \\
 \midrule 
$SUBST^{n,MGPM}_{m, c}$ & Notional substitution effect between the margin-making sector $m$ and the over margin-makings sectors $mm$ for the imported commodity $c$ \\
 \midrule 
$SUBST^{n,MGRM}_{c}$ & Notional substitution effect induced by a change in the relative price between imported and domestically produced commodity $c$ for margins received \\
 \midrule 
$SUBST^{n,X}_{c}$ & Notional substitution effect induced by a change in the relative price between export prices and (converted in domestic currency) international prices for the commodity $c$ \\
 \midrule 
$SUBST^{n,XM}_{c}$ & Notional substitution effect induced by a change in the relative price between imported and domestic products $c$ for exports \\
 \midrule 
$SUBST^{X}_{c}$ & Substitution effect on the exports of the commodity $c$ \\
 \midrule 
$SUBST^{XM}_{c}$ & Substitution effect on the government final consumption for the imported commodity $c$ \\
 \midrule 
$TRSP_{s}$ & Demand for transport commodities by sector $s$ \\
 \midrule 
$Un$ & Unemployment \\
 \midrule 
$UnR$ & Unemployment rate \\
 \midrule 
$VA$ & Aggregate value-added \\
 \midrule 
$VA_{s}$ & Value-added of sector $s$ \\
 \midrule 
$VA^{VAL}_{s}$ & Value-added of sector $s$ expressed in value \\
 \midrule 
$W$ & Average wage \\
 \midrule 
$W_{s}$ & Wages of the sector $s$ \\
 \midrule 
$W^{n}_{s}$ & Notional wage in sector $s$ \\
 \midrule 
$WAGES$ & Aggregate gross wages paid by sectors \\
 \midrule 
$WAGES_{s}$ & Gross wages paid by sector $s$ including employees (but not employers)' social contribution \\
 \midrule 
$WAPop$ & Working-age population \\
 \midrule 
$X$ & Aggregate exports, expressed at market price \\
 \midrule 
$X_{c}$ & Foreign demand for exports of commmodity $c$ \\
 \midrule 
$XD$ & Aggregate domestically produced exports, expressed at market price \\
 \midrule 
$XD_{c}$ & Exports of domestically produced commodity $c$ \\
 \midrule 
$XM$ & Aggregate imported exports (re-exports), expressed at market price \\
 \midrule 
$XM_{c}$ & Exports of imported commodity $c$ \\
 \midrule 
$Y$ & Aggregate production, expressed at basic price \\
 \midrule 
$Y_{c, s}$ & Production of commodity $c$ by sector $s$ \\
 \midrule 
$Y_{s}$ & Production of sector $s$, expressed at basic price \\
 \midrule 
$Y^{e}_{s}$ & Expected production \\
 \midrule 
$YCAP_{s}$ & Production capacity of the sector $s$ \\
 \midrule 
$YQ$ & Domestic production, expressed at basic price \\
 \midrule 
$YQ_{c}$ & Production of commodity $c$, expressed at basic price \\
 \midrule 
$YQ^{bis}_{c}$ & Production of commodity $c$, expressed at basic price (for verification) \\
 \midrule 
$YQS_{c}$ & Quantity of domestically produced commodity $c$ expressed at selling price \\
\end{longtable}
\end{document}
